\documentclass[titlepage,a4paper]{jsarticle}
\usepackage{../../../sty/import}% 各種パッケージインポート
\usepackage{../../../sty/title}% タイトルページの変更
\renewcommand{\thesection}{問\arabic{section}}
\setcounter{section}{10}
%% タイトルページの変数
% レポートタイトル
\title{7月4日課題}
% 提出日
\expdate{\today}
% 科目名
\subject{データベースと応用システム}
% 分野
\class{情報経営システム工学分野}
% 学年
\grade{B4}
% 学籍番号
\mynumber{24336488}
% 記述者
\author{本間三暉}

\begin{document}
% titleページ作成
\maketitle

\section{WAL(Write Ahead Logging)プロトコルについて説明しなさい.}
WAL(Write Ahead Logging)プロトコルとは,データベースの更新処理を行う前に,その更新内容をログファイルに先に記録する方式である.これにより,障害が発生した場合でも,ログを用いてデータベースを復旧することが可能になる.

更新の手順は以下の通りである.
\begin{enumerate}
  \item \texttt{begin transaction} レコードをログに記録
  \item 更新前のデータをログに記録
  \item 更新後のデータをログに記録
  \item データベース本体を更新
  \item \texttt{commit} レコードをログに記録
  \item \texttt{end transaction} レコードをログに記録
\end{enumerate}
\section{あるデータベースでは,毎週日曜日にフルバックアップ,それ以外の日に増分バックアップを実施している.障害が発生して火曜日時点のデータベースを復元したい場合,必要となる作業手順を説明しなさい.}
毎週日曜日にフルバックアップ,それ以外の日に増分バックアップを実施している環境で,火曜日時点のデータベースを復元するには,次の手順を順に実行する.
\begin{enumerate}
  \item 日曜日に取得したフルバックアップをリストアする.
  \item 月曜日の増分バックアップを適用する.
  \item 火曜日の増分バックアップを適用する.
  \item 必要に応じてログファイルを用いてロールフォワードを行い,火曜日終了時点まで復旧する.
\end{enumerate}
増分バックアップは「前回のバックアップとの差分」を保持するため,フルバックアップ→前日の増分→当日の増分の順に適用する必要がある.

\section{トランザクション障害および発生時の復旧方法について説明しなさい.}
トランザクション障害は,メモリ不足,デッドロック,通信障害などによりトランザクションが途中で異常終了することで発生する.復旧方法は,ログファイルを用いてそのトランザクションが行った更新操作をすべて取り消すロールバックである.

\section{システム障害とは何か説明しなさい.また,チェックポイント後にコミットが完了してその後障害が発生しているデータの復旧方法について説明しなさい. }
システム障害とは,DBMS のクラッシュ,OS 障害,電源断など,データベースシステム全体が停止する障害である.復旧時には次の対応を行う.
\begin{itemize}
  \item チェックポイント以前にコミット済みのデータ:ディスクに書き込まれているため復旧不要.
  \item チェックポイント後にコミットが完了しているデータ:ログを用いてロールフォワード(redo)する.
  \item コミットが完了していないデータ:ロールバック(undo)して破棄する.
\end{itemize}
以上により,一貫性を保った状態でデータベースを再起動できる.

\section{メディア障害および発生時の復旧方法について説明しなさい.}
メディア障害は,データベースを格納しているディスク装置自体が物理的に故障し,データが読み出せなくなる障害である.対処方法は,直近のフルバックアップをリストアした後,ログファイルを用いてロールフォワードを行い,障害発生直前の状態へ復旧することである.
\end{document}