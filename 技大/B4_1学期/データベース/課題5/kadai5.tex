\documentclass[titlepage,a4paper]{jsarticle}
\usepackage{../../../sty/import}% 各種パッケージインポート
\usepackage{../../../sty/title}% タイトルページの変更
\renewcommand{\thesection}{問\arabic{section}}
\setcounter{section}{15}
%% タイトルページの変数
% レポートタイトル
\title{7月11日課題}
% 提出日
\expdate{\today}
% 科目名
\subject{データベースと応用システム}
% 分野
\class{情報経営システム工学分野}
% 学年
\grade{B4}
% 学籍番号
\mynumber{24336488}
% 記述者
\author{本間三暉}

\begin{document}
% titleページ作成
\maketitle

\section{分散データベースにおいて,垂直分散と水平分散のそれぞれの利点と欠点を説明しなさい.}
\begin{table}[h]
\centering
\begin{tabular}{|l|p{6cm}|p{6cm}|}
\hline
分散方式 & 利点 & 欠点 \\ \hline
垂直分散(マスター+レプリカ) &
\begin{itemize}\item 更新経路が単純で設計が容易である.\item 読み取り負荷をレプリカに逃がせるため参照性能を高めやすい.\end{itemize} &
\begin{itemize}\item 書き込みがマスターに集中しボトルネックになる.\item マスター障害時に全体停止へつながりやすい.\end{itemize} \\ \hline
水平分散(シャーディング) &
\begin{itemize}\item ノード追加で容易にスケールアウトできる.\item ノード障害が局所化し可用性が高い.\end{itemize} &
\begin{itemize}\item データ分割規則と問い合わせ最適化が複雑である.\item 複数ノードをまたぐトランザクションの整合性維持が難しい.\end{itemize} \\ \hline
\end{tabular}
\end{table}

\bigskip
垂直分散はシンプルだが単一マスターに依存する.水平分散は拡張性と可用性が高いが設計と運用が複雑である.ワークロードと可用性要件を踏まえて適切に選択すべきである.

\section{分散データベースの透過性について説明し,透過性の要素を列挙しなさい.}

\section{分散データベースの1相コミットメント制御の問題点と2相コミットメント制御の利点を説明しなさい.}

\section{銀行の残高を記録する分散データベースをレプリケーションにより負荷分散する場合,望ましい方法を説明しなさい.}


\end{document}