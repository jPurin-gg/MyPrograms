\documentclass[titlepage,a4paper]{jsarticle}
\usepackage{../../../sty/import}% 各種パッケージインポート
\usepackage{../../../sty/title}% タイトルページの変更
\renewcommand{\thesection}{問\arabic{section}}
\setcounter{section}{15}
%% タイトルページの変数
% レポートタイトル
\title{7月11日課題}
% 提出日
\expdate{\today}
% 科目名
\subject{データベースと応用システム}
% 分野
\class{情報経営システム工学分野}
% 学年
\grade{B4}
% 学籍番号
\mynumber{24336488}
% 記述者
\author{本間三暉}

\begin{document}
% titleページ作成
\maketitle

\section{分散データベースにおいて,垂直分散と水平分散のそれぞれの利点と欠点を説明しなさい.}
\subsection*{垂直分散(マスター+レプリカ)}

\textbf{利点}
\begin{itemize}
  \item 更新経路が単純で設計が容易である.
  \item 読み取り負荷をレプリカに逃がせるため参照性能を高めやすい.
\end{itemize}

\textbf{欠点}
\begin{itemize}
  \item 書き込みがマスターに集中しボトルネックになる.
  \item マスター障害時に全体停止につながりやすい.
\end{itemize}

\subsection*{水平分散(シャーディング)}

\textbf{利点}
\begin{itemize}
  \item ノード追加で容易にスケールアウトできる.
  \item ノード障害が局所化し,可用性が高い.
\end{itemize}

\textbf{欠点}
\begin{itemize}
  \item データ分割規則と問い合わせ最適化が複雑である.
  \item 複数ノードをまたぐトランザクションの整合性維持が難しい.
\end{itemize}
\section{分散データベースの透過性について説明し,透過性の要素を列挙しなさい.}
透過性とは,利用者に分散を意識させず単一のデータベースとして見せる性質である.具体的な要素は次の七つである.  
\begin{enumerate}
  \item 位置透過性:データ保存場所を意識させない.
  \item 移動透過性:データ移動がアプリケーションに影響しない.
  \item アクセス透過性:ノード差異なく同一 API で操作できる.
  \item 複製透過性:複数コピーの存在を隠蔽する.
  \item 分割透過性:表や行列の分割配置を意識させない.
  \item 障害透過性:一部ノード障害時もサービスを継続できる.
  \item 規模透過性:ノード増減が利用者に影響しない.
\end{enumerate}

これらを高水準で満たすことで,アプリケーション開発者は分散構成を意識せずにロジック実装へ集中できる.

\section{分散データベースの1相コミットメント制御の問題点と2相コミットメント制御の利点を説明しなさい.}
\subsection*{1相コミットメント制御(1PC)の問題点}
単一メッセージで各ノードにコミットを指示する方式である.途中で従ノードが障害を起こすと,他ノードとの間でコミット状態が不一致となり原子性が損なわれる.

\subsection*{2相コミットメント制御(2PC)の利点}
\begin{enumerate}
  \item \textbf{準備フェーズ}:コーディネータが各ノードにコミット可能かを問い合わせ,ノードはロックを保持したまま ``準備完了'' か ``中止'' を応答する.
  \item \textbf{決定フェーズ}:全ノードが準備完了の場合にのみコミットを一斉指示し,いずれかが中止なら全ノードでアボートする.
\end{enumerate}
この二段階により,全サイトで同一結果(コミット/アボート)となり原子性を保証できる.ただしコーディネータ障害時にはロック保持時間が長引くため,タイムアウト処理や3相コミットなどの補助策が検討される.

\section{銀行の残高を記録する分散データベースをレプリケーションにより負荷分散する場合,望ましい方法を説明しなさい.}
銀行残高は強整合性が必須である.したがって次の構成が望ましい.  

\begin{itemize}
  \item 同期レプリケーションを採用し,書き込みをマスターで受け付けつつ全レプリカへ即時反映する.
  \item 読み取り処理はレプリカへ振り分け,読み込み性能を向上させる.
  \item マスター障害時は自動フェイルオーバで新マスターを昇格させる.
  \item トランザクション分離レベルを厳格に設定し,ログシッピングや監査証跡を併用して可用性と信頼性を高める.
\end{itemize}

この構成であれば,読み取り負荷を分散しながら書き込みの一貫性と可用性を両立できる.

\begin{thebibliography}{99}
  \bibitem{}データベースと応用システムの講義資料
\end{thebibliography}
\end{document}