\documentclass[titlepage,a4paper]{jsarticle}
\usepackage{../../../sty/import}% 各種パッケージインポート
\usepackage{../../../sty/title}% タイトルページの変更
\renewcommand{\thesection}{問\arabic{section}}
\setcounter{section}{6}
%% タイトルページの変数
% レポートタイトル
\title{6月27日課題}
% 提出日
\expdate{\today}
% 科目名
\subject{データベースと応用システム}
% 分野
\class{情報経営システム工学分野}
% 学年
\grade{B4}
% 学籍番号
\mynumber{24336488}
% 記述者
\author{本間三暉}

\begin{document}
% titleページ作成
\maketitle
\section{デッドロックおよびその検出・回避法ついて説明しなさい.}

\textbf{デッドロック}とは,複数のトランザクションが互いに必要なリソースをロックしてしまい,処理が互いに進行できなくなる状態を指す.
デッドロックとは,複数のトランザクションが互いに必要なリソースをロックしてしまい,処理が互いに進行できなくなる状態を指す.デッドロックへの対策は,一般的に「検出と回復」「防止」といったアプローチに分類される.

\subsection*{検出と回復 (Detection and Recovery)}
デッドロックの発生を許容し,発生を検出した際に回復処理を行う方法.
\begin{itemize}
    \item \textbf{待ちグラフ方式}: トランザクション間のリソースの待ち関係を有向グラフで表し,閉路(サイクル)が存在するかを確認することでデッドロックを検出する .
    \item \textbf{タイムアウト方式}: 一定時間以上ロック待ち状態にあるトランザクションをデッドロック状態とみなし,強制的にロールバックさせて回復する実用的な方法 .
\end{itemize}

\subsection*{防止 (Prevention)}
デッドロックが発生する条件をなくすことで,デッドロックを未然に防ぐ方法.
\begin{itemize}
    \item \textbf{リソースのロック順序を固定する}: 全てのトランザクションが,予め定められた同じ順序でリソースをロックするようにすることで,循環待ちを発生させない .
    \item \textbf{トランザクション開始時に一括ロックする}: トランザクションの実行前に,必要なリソースをすべて一括でロックする.これにより,リソースを保持したまま他のリソースを待つ状況を防ぐ .
    \item \textbf{時刻印アルゴリズム}: トランザクションに時刻印(タイムスタンプ)を割り当て,その新旧に基づいて優先度を決定する.競合が発生した際に,新しいトランザクションをロールバックさせるなどのルールを設けることで,デッドロックを防ぐ .
\end{itemize}

\subsection*{その他のアプローチ}
\begin{itemize}
    \item \textbf{楽観的排他制御}: ロックを用いずに処理を進め,最終的な更新(コミット)前に,他のトランザクションによる更新と競合がなかったかを確認する方式.競合があった場合は,トランザクションをロールバックする.ロックをかけないため,デッドロックは原理的に発生しない .
\end{itemize}

\section{ダーティリードとそれを回避するための方法について説明しなさい.}

ダーティリードとは,他のトランザクションによって更新されたが,まだコミットされていないデータを読み取ってしまう現象である .この場合,読み取り元のトランザクションがロールバックされると,読み取ったデータは存在しなかったことになり,不整合が発生する可能性がある .

最も低い隔離レベルである「Read Uncommitted」ではダーティリードが発生しうる.この問題を回避するには,「Read Committed」以上の隔離レベルを使用することで,コミットされていないデータの読み取りを禁止する必要がある .

\section{ノンリピータブルリードとファントムリードについて説明しなさい.}

\begin{itemize}
  \item \textbf{ノンリピータブルリード}:同じトランザクション内で同じデータを2回読み取った際に,他のトランザクションによって更新されたために読み取る値が異なる現象.
  \item \textbf{ファントムリード}:ある条件で検索した結果セットに対し,他のトランザクションによってレコードの追加・削除が行われることで,2回目の検索時に異なる結果が返る現象.
\end{itemize}

\section{隔離性水準におけるserializableレベルについて説明しなさい.}

\textbf{Serializable}は,トランザクションの隔離性水準の中で最も厳格なレベルであり,すべてのトランザクションを直列に実行したかのような結果を保証する.

このレベルでは以下の3つの問題をすべて防止する:
\begin{itemize}
  \item ダーティリード
  \item ノンリピータブルリード
  \item ファントムリード
\end{itemize}

一方で,ロックが多く発生するため,スループットの低下やロック待ちなどの影響が大きくなる可能性がある.

% 参考文献(Web上の資料)
\begin{thebibliography}{99}
  \bibitem{}データベースと応用システムの講義資料
\bibitem{atmarkit_commit}
@IT, 「コミット」「ロールバック」とは?:用語解説,\\
\url{https://www.atmarkit.co.jp/ait/articles/1703/01/news195.html}

\bibitem{cloudear_deadlock}
Cloudearブログ, デッドロックの検出と回避の解説,\\
\url{https://cloudear.jp/blog/?p=1335}
\end{thebibliography}
\end{document}