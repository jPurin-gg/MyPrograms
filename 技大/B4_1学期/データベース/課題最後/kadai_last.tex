\documentclass[titlepage,a4paper]{jsarticle}
\usepackage{../../../sty/import}% 各種パッケージインポート
\usepackage{../../../sty/no_titlepage}% タイトルページの変更
\geometry{top=15mm,bottom=15mm,left=15mm,right=15mm}
\usepackage{multicol}
\renewcommand{\thesection}{課題\arabic{section}}

%% タイトルページの変数
% レポートタイトル
\title{データベースと応用システム最終レポート}
% 提出日
\expdate{\today}
% 科目名
\subject{データベースと応用システム}
% 分野
\class{情報経営システム工学分野}
% 学年
\grade{B4}
% 学籍番号
\mynumber{24336488}
% 記述者
\author{本間三暉}

\begin{document}
% titleページ作成
% \maketitle
\section{あなたが日頃よく用いるデータベースシステムを一つ挙げ,その利点と問題点を簡潔に論じなさい.}\label{kadai1}
私は日頃よく用いるデータベースシステムとして,X(旧 Twitter)を挙げる.
X(旧Twitter)は世界中の投稿を即時に検索でき,発信者とも気軽に対話できるうえ,AIが興味に合ったツイートを選んでくれるため,必要な情報を効率よく得られる.
しかし同じ仕組みがフィルターバブルを生み,誤情報やデマが拡散しやすいという副作用もある.
便利さを保ちつつ偏りと誤情報を防ぐためには,アルゴリズムの透明性と検証体制の強化が欠かせない.

\section{\ref{kadai1}で述べた問題点を解決する方法を論じなさい.}
X には,関心が近い投稿だけが並んで視野が狭くなる「フィルターバブル」と,誤った情報が一気に広がる「デマ拡散」という二つの問題がある.
これらはどちらも,利用者の反応を重視して投稿を順位づけする仕組みから生まれている.
したがって,順位づけの考え方を少し変えるだけで,両方をいくらか抑えられるはずである.

まずフィルターバブルについては,タイムラインに「違う立場の投稿」を少量まぜる方法が有効だと報告されている\cite{filter-algo}.
おすすめを作る計算で“関心度”だけでなく“話題の違い”にも点数を付けると,似た内容ばかりが続く状態が大きく減ったという結果である.
このやり方を実装する際は,利用者が画面上のスライダで「混ぜる割合」を自分で調整できるようにするとよい.
強制ではなく選択肢として示すことで,サービスの透明性とユーザの納得感を高められるからである.

次にデマ拡散対策としては,投稿やユーザ同士のつながりをまるごと学習する AI を用い,怪しい投稿を数秒以内に見つける仕組みが有望である\cite{gnn-deim2024}.
検知した投稿には「検証中」のラベルを付けて一時的に表示を減らし,同時に外部のファクトチェック機関に自動で照会する.
もし誤情報だと判定された場合は,その投稿を警告付きで残すか削除し,悪意あるアカウントは凍結する.

こうした重い AI 計算は,専用の GPU サーバに分けて実行すれば,検索や投稿のスピードを落とさずにすむ.
また,各投稿に「なぜ勧められたか」「なぜ疑われたか」を短い説明文で示すことで,判断の理由を利用者が確認できる.
多様化アルゴリズムとリアルタイム検知を組み合わせ,その過程を開示する――この三点をそろえることで,X は便利さを保ちつつ,片寄りと誤情報の問題を大きく減らせると考える.

\begin{thebibliography}{99}
\bibitem{text} データベースと応用システムの講義資料
\bibitem{filter-algo} 金子悠太ほか 「SNS におけるフィルターバブル改善を目的としたアルゴリズムの研究」 社会システム情報学会研究発表会 (2020) \url{https://journals.socsys.org/symposium022/pdf/022-044.pdf}
\bibitem{gnn-deim2024} 太田航太ほか 「異種グラフ表現学習を用いた悪意のある SNS 投稿の検出」 DEIM2024 (2024) \url{https://confit.atlas.jp/guide/event-img/deim2024/T3-C-9-01/public/pdf}
\end{thebibliography}
\end{document}
%% レポートの上部に「学籍番号,所属分野・研究室,氏名」を記載すること.
%% 課題1と課題2を合わせて,A4用紙1枚以内(図表や参考文献を含む)とすること.
%% フォントサイズは10.5pt以上とすること.太字・下線や色を用いてよい.可読性があれば,フォント・文字幅・行間は指定しない.
%% 提出時のファイル名は
%% "学籍番号"_report.pdf
%% とすること