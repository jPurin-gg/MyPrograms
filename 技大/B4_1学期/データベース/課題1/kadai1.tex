\documentclass[titlepage,a4paper]{jsarticle}
\usepackage{../../../sty/import}% 各種パッケージインポート
\usepackage{../../../sty/title}% タイトルページの変更
\renewcommand{\thesection}{問\arabic{section}}
%% タイトルページの変数
% レポートタイトル
\title{6月13日課題}
% 提出日
\expdate{\today}
% 科目名
\subject{データベースと応用システム}
% 分野
\class{情報経営システム工学分野}
% 学年
\grade{B4}
% 学籍番号
\mynumber{24336488}
% 記述者
\author{本間三暉}

\begin{document}
% titleページ作成
\maketitle
\section{主記憶と二次記憶のそれぞれについて説明しなさい.}
\begin{itemize}
  \item \textbf{主記憶(メインメモリ)}:半導体メモリで構成され,CPUが直接アクセス可能な記憶装置である.揮発性であり,電源を切ると内容が消える.アクセス速度は高速(ナノ秒オーダ)だが,容量は数GB程度と比較的小さい.
  \item \textbf{二次記憶(補助記憶)}:主にハードディスクなどの不揮発性記憶装置で構成され,CPUから直接アクセスできないためアクセス速度は遅い(ミリ秒オーダ).一方で容量は大きく,数GB〜数TBに及ぶ.
\end{itemize}

\section{代表的なインデックスとしてB木が挙げられる.2分木の場合の検索時間について説明しなさい.}
2分木(バイナリツリー)の場合,データ数を $n$ とすると,木の高さは $\log_2(n)$ となるため,検索時間の計算量は $O(\log n)$ となる.  
つまり,データ数が2倍になっても,検索ステップ数は1回増えるだけで済み,効率的に探索が可能である.  
このように,検索時間の増加はデータ数に対して対数オーダであり,大規模データでも高速な検索が可能となる.

\section{B木インデックスの欠点を説明した上で,B+木インデックスの利点を説明しなさい.}
\begin{itemize}
  \item \textbf{B木インデックスの欠点}:B木では,データが内部ノードにも格納されるため,順次アクセス(レンジクエリなど)が非効率である.順にデータをたどる際に何度も木構造を経由する必要がある.
  \item \textbf{B+木インデックスの利点}:B+木では,全てのデータをリーフノードに格納し,それらをポインタで連結することで順次アクセスを効率化している.B木のランダムアクセスの利点を保持しつつ,レンジクエリや順次検索に強いため,多くのデータベースで採用されている.
\end{itemize}

\section{ハッシュインデックスの仕組みと検索効率について説明しなさい.}
\begin{itemize}
  \item \textbf{仕組み}:ハッシュインデックスでは,キーにハッシュ関数を適用し,そのハッシュ値によって格納位置を決定する.この方法により,特定のキーに対して直接アドレスを求めてアクセスできるため,検索が高速になる.
  \item \textbf{検索効率}:検索時間はデータベースの規模に依存せず,理論上 $O(1)$(定数時間)で高速である.ただし,異なるキーが同じハッシュ値を持つ衝突(コンフリクト)が発生する可能性があり,この場合はオープンハッシュ法やチェイン法などで対処する.また,範囲検索や部分一致検索には適さないため,使用用途には制限がある.
\end{itemize}
% 参考文献
\begin{thebibliography}{99}
\bibitem{}データベースと応用システムの講義資料
\end{thebibliography}

\end{document}