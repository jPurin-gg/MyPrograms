\documentclass[titlepage,a4paper]{jsarticle}
\usepackage{../../sty/import}% 各種パッケージインポート
\usepackage{../../sty/title_kyoudou}% タイトルページの変更

%% タイトルページの変数
% レポートタイトル
\title{量子コンピュータプログラミング}
% 提出日
\expdate{\today}
% 科目名
\subject{情報システム工学実験}
% 分野
\class{情報経営システム工学分野}
% 学年
\grade{B3}
% 学籍番号
\mynumber{24336488}
% 記述者
\author{本間三暉}
% グループ名 % もし班があるやつならtitle_team.styを入れる
\coauthor{%
{学籍番号:}22100289 & {氏名:}浅野 繭\\
{学籍番号:}22105590 & {氏名:}筒井 翼\\
{学籍番号:}22100986 & {氏名:}板山修大\\
}
%
%記載例:
%\coauthor{%
%  2番 & 新潟 花子\\
%  11番 & 三条 次郎}
%%
\begin{document}
% titleページ作成
\maketitle
\section{目的}
% ・今回の実験について、自分なりに設定した目的を記述して下さい。

\section{原理}
% ・量子論理回路、グローバーのアルゴリズムについて説明してください。
\section{論理パズル}
% ・グループで考えた論理パズル2人用、3人用、4人用
% についてわかりやすく記述してください。
% ・自分の担当部分を明記してください。
\section{パズルの論理式}
% ・グループで考えたパズルの論理式についてわかりやすく記述してください。
% ・自分の担当部分を明記してください。
\section{量子回路}
$\overline{A \cup B}$
% ・パズルを解く回路を記載し、その回路について説明してください。
% ・自分の担当部分を明記してください。
\section{実行結果}
% ・QCEngine とIBMQ のシミュレータでの結果を掲載してください。
\section{考察}
% ・実験結果について考察してください。
\section{感想}
% ※ここは採点の対象外です。
% 今回の実験の内容について感想などあれば記述して下さい。
% 次年度以降の実験の実施に役立てたいと思います。
レポートを書く際に結果を使ったりするのならば,ソースコードを保存した方が良いなど実験中に言ってくれると嬉しかった.
\end{document}
