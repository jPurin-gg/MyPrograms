\documentclass[titlepage,a4paper]{jsarticle}
\usepackage{../../../sty/import}% 各種パッケージインポート
\usepackage{../../../sty/title_team}% タイトルページの変更

%% タイトルページの変数
% レポートタイトル
\title{}
% 提出日
\expdate{\today}
% 科目名
\subject{産学連携実践的AI応用}
% 分野
\class{情報経営システム工学分野}
% 学年
\grade{B3}
% 学籍番号
\mynumber{24336488}
% 記述者
\author{本間 三暉}
% グループ名 % もし班があるやつならtitle_team.styを入れる
\team{3}
% 共同実験者 % もし共同実験者が必要なやつならtitle_kyoudou.styを入れる
\coauthor{%
\textbf{学籍番号:22100986} & \textbf{氏名:板山 修大}\\
\textbf{学籍番号:22105590} & \textbf{氏名:筒井 翼}\\
\textbf{学籍番号:22106685} & \textbf{氏名:花田 光}\\
\textbf{学籍番号:22100289} & \textbf{氏名:浅野 繭}\\
}

\begin{document}
% titleページ作成
\maketitle
\section{データの観測構造をモデル化する}
データの観測構造を明らかにするため,顧客データに対して以下の分析を実施した.
まず,デモグラフィック分析として顧客の性別割合と年代別の顧客数を可視化し,ターゲットセグメントを把握した.
次に,RFM分析を用いて購買行動を分析し,顧客のロイヤルティレベルを数値的に評価した.
さらに,クーポン利用データを詳細に解析し,クーポン施策が顧客行動に与える影響を定量的に把握した.
これらの結果をもとに,優良顧客に対するロイヤルティクーポン,離脱リスク顧客に対するカムバッククーポンの仮説を設定しようとした.
\section{解くべき問題を特定する}
観測データの分析結果から,売上最大化のためには顧客セグメントごとに適切なクーポン発行条件を最適化する必要があることを特定した.
そこで,本研究ではLightGBMを用いてクーポン発行条件をモデル化し,売上金額を目的変数,クーポン利用情報や購買頻度,累計割引額などを特徴量として設定した.
このモデルにより,最適な発行条件の特定を目指した.
\section{観測データのみを用いて問題を解く方法を考える}%%%%
基礎分析を行うことでそこから仮説を建てた上で,それに合わせてパラメータを変更することを考える.
私の班ではあまり行わなかったので,他の班を参考にして記述する.
最適なクーポン発行条件を導出した.

\subsection{クーポンの役割と目的}
No.6の戦略では,クーポンが顧客の購買頻度やサイズ選択に与える影響を解析し,購買頻度を高めるとともにサイズアップを促進することを目的としていて,クーポンは以下の2種類に分類されると仮定していた.
\begin{itemize}
  \item \textbf{クーポン1}:ロイヤルティクーポン.高頻度来店者に対し,来店サイクルを短縮する効果を持つ.
  \item \textbf{クーポン2}:カムバッククーポン.低頻度来店者や休眠顧客を対象とし,顧客ステージの引き上げを目的とする.
\end{itemize}

\subsection{データ分析の手法}
観測データから以下の条件を抽出し,来店頻度と売上への影響を解析した:
\begin{itemize}
  \item 来店頻度:クーポン利用者の平均再来店日数(クーポン1では2.26日,クーポン2では22.16日)
  \item 購入サイズの分布:Sサイズ38.06\%,Mサイズ29.77\%,Lサイズ32.16\%
  \item 顧客属性:性別,年齢層,購買傾向
\end{itemize}

また,購買関数を仮定し,個人売上の期待値を以下の式で推定した:
\[
  P_{p} = \{ \text{個人来店確率} + \text{クーポンの影響度} \} \times \{ \text{サイズ期待値} \}
\]
これを基に,クーポン発行条件が売上に与える影響をシミュレーションした.\\


このような方法でNo.6は売上一位に輝いた.

\section{機械学習モデルを学習する or 統計分析手法を適用する}

ロイヤルティクーポンの最適化においては,LightGBMモデルを採用し,グリッドサーチを活用してハイパーパラメータの調整を行った.特徴量には,クーポン利用履歴,来店頻度,累計割引額,最終購入日からの経過日数などを用い,売上金額を目的変数とした.特に,ロイヤルティクーポンを利用する顧客層において,来店サイクルを短縮する最適な期間と購入回数を探索した.このモデルは,顧客ごとの購買行動を予測し,期間や購入回数を最適化することで,売上最大化を目指すものである.

一方,カムバッククーポンでは確率モデルを採用し,クーポンが来店確率に与える影響を定量化した.このモデルは,顧客の再来店確率を予測するために,年齢,性別,過去の来店頻度などの要因を取り入れたものである.また,グリッドサーチを活用し,割引率,発行日数,対象サイズなどのパラメータを調整した.確率モデルは,来店確率低下のメカニズムを解析し,カムバッククーポンが再来店を促進する条件を明らかにするために用いられた.

しかし,これらのモデルに基づく予測値や出力結果について,現実の購買行動と乖離している可能性が示唆された.モデル精度の評価に用いた指標や交差検証手法の妥当性が十分に検証されておらず,導き出された数値が信頼に足るものか疑問が残る.さらに,モデル構築において,因果関係の定量的な検証が不十分であったことが課題として浮上した.

\section{施策を導入する}

モデルに基づき,ロイヤルティクーポンとカムバッククーポンの発行条件を設定し,これを実際に適用した.ロイヤルティクーポンでは,期間を5日間,購入回数を2回とした.この条件は,クーポン利用顧客の来店頻度を高めることを目的としている.一方,カムバッククーポンでは,発行条件として日数を14日,割引率を5\%,対象サイズをMに設定した.この設定は,休眠顧客や低頻度来店者を再来店に導くことを狙ったものである.

施策実施に際して,ターゲット顧客へのクーポン配布数や配布方法,クーポンの使用条件を詳細に設計し,その効果をリアルタイムで測定した.ただし,クーポン配布のタイミングや発行数に関する詳細な戦略の欠如が,施策全体の効果を制限する要因となった可能性がある.

\section{施策導入後の分析}

施策導入後の分析の結果,売上が目標に達しなかった原因として,以下の点が挙げられる:
\begin{itemize}
  \item ロイヤルティクーポンの発行期間が短すぎたため,既存顧客の購買行動を効果的に変化させられなかった.
  \item カムバッククーポンのターゲット設定が不十分であり,休眠顧客の再来店を十分に促進できなかった.
  \item クーポン利用数は増加したものの,利益率が低下し,売上増加にはつながらなかった.
  \item LightGBMモデルや確率モデルが予測した数値が,現実の購買行動を反映しておらず,その根拠や妥当性に欠けていた可能性がある.
\end{itemize}

特に,モデルにより導き出された割引率や発行条件が,現実の購買データに基づく効果を十分に検証されていなかったことが問題点として挙げられる.また,顧客層の購買行動に関する仮説の検証が不十分であり,観測データとの整合性に課題が残った.

今後は,モデル構築時にデータ前処理や交差検証手法を見直し,モデルの精度を向上させる必要がある.さらに,因果推論を用いた分析を取り入れ,クーポンの効果をより正確に測定することが求められる.加えて,施策導入後のデータを基に,A/Bテストや仮説検証を繰り返し,最適な条件を継続的に模索するアプローチが必要である.

\section{他のグループ発表を踏まえた比較と考察}

各グループが発表した戦略を分析した結果,成功した戦略と効果が限定的だった戦略の間には明確な差異が認められた.
成功した戦略の共通点として,ターゲットの明確化と割引率の適正な設定が挙げられる.
特に,\textbf{No.6の戦略}では,ターゲット層を20代・30代女性に絞り,サイズLに対して割引率5\%を設定することで,売上が最大となる12,109,195円を達成した.
この戦略の成功要因は,顧客ニーズを的確に捉えたターゲティングと,利益を損なわない割引率のバランスにあると考えられる.

一方で,効果が限定的だった戦略では,ターゲットの絞り込みが不足していた点が共通して見られる.
例えば,\textbf{No.3の戦略}では全対象属性に一律5\%割引を提供したものの,クーポンの効果が分散し,売上は11,176,480円と伸び悩んだ.
また,\textbf{No.12の戦略}では,最終購入から12日以上経過した顧客を対象にMサイズ限定で10\%割引クーポンを発行したが,ターゲットが限定的すぎたため顧客の購買行動を大きく促進するには至らなかったと推測される.

さらに,\textbf{No.11の戦略}では,最終購入から30日以上経過した全会員を対象に全サイズ30\%オフのクーポンを提供し,11,740,550円という高い売上を達成した.
ただし,この戦略は広範な顧客を対象にすることで売上を確保した一方,割引率が高いため利益率に与える影響が懸念される.
利益を最大化するためには,割引率の適正化や対象顧客の絞り込みがさらに求められるであろう.

これらの結果を踏まえると,成功した戦略の特徴として以下の要素が重要であることが明らかになった.
\begin{enumerate}
  \item \textbf{ターゲットの明確化}: 成功した戦略では,ターゲットとなる顧客層が具体的に設定されており,施策の効果が集中していた.
  \item \textbf{割引率の適正化}: 過剰な割引率は利益を圧迫し,低すぎる割引率は顧客の関心を引きにくい傾向が見られた.
  \item \textbf{サイズや商品の選定}: No.6のように特定の商品(サイズL)を対象とすることで,ターゲット層の購買意欲を効率的に高めることができた.
\end{enumerate}

今後の改善策としては,成功した戦略の要素を他の戦略に応用することが有効である.
例えば,データ分析を基に顧客セグメントをさらに詳細に分け,ターゲティングの精度を向上させることが重要である.
また,A/Bテストを実施して割引率や対象商品の最適な組み合わせを見出すことが必要である.
これにより,顧客満足度を高めながら利益率を向上させる施策が期待できる.

\section{まとめと今後の課題}
私達の班では,クーポン発行条件の最適化を目指してデータ分析とモデル構築を行おうとした.
しかし,施策導入後の売上減少という結果から,クーポン発行条件や目的変数の再検討が必要である.

機械学習などを用いて売上を最適化しようとしたが,班内での理解が浅かったことや,ChatGPTに頼りすぎてしまったことから,
それっぽいだけのものを作成し,先生に指摘されるまで気づかなかったことが一番の反省として挙げられる.
AIを上手く活用することも大切だが,AIに使われることのないよう,より一層気をつけていきたい.


また,根本的に個人的な時間が足りていないことも原因として挙げられるため,自らの時間管理も徹底していきたいと考えた.
\section{授業に関する感想と要望}
もう少しにグループワークに関係する部分を長くしてほしい.
後半かなり少ない時間でやることになってしまい,話し合う時間もあまり取れなかった.
また,せっかく卒研発表のような空気が味わえるなら,中間発表のようなものを行ったりして,方向性が間違っていた場合の修正をしてもらえると嬉しいと感じた.


% % 参考文献
% \begin{thebibliography}{99}
%   \bibitem{}
% \end{thebibliography}

\end{document}