\documentclass[titlepage,a4paper]{jsarticle}
\usepackage{../../../sty/import}% 各種パッケージインポート
\usepackage{../../../sty/title}% タイトルページの変更

%% タイトルページの変数
% レポートタイトル
\title{国家・資本家・労働者の関係性について}
% 提出日
\expdate{\today}
% 科目名
\subject{社会福祉概論}
% 分野
\class{情報経営システム工学分野}
% 学年
\grade{B3}
% 学籍番号
\mynumber{24336488}
% 記述者
\author{本間三暉}

\begin{document}
% titleページ作成
\maketitle
\section{語句の定義}
まず,国家,資本家,労働者についてざっくりと定義付けを行う.
コトバンクより引用すると,
``国家とは一定の領土とそこに居住する人々からなり、統治組織をもつ政治的共同体。または、その組織・制度。主権・領土・人民がその3要素とされる。\cite{国家}''
``資本家とは企業に資本を提供している者。経営を直接に担当している機能資本家と、単に利益の配分にあずかるだけの無機能資本家とに分類することもできる。\cite{資本家}''
``労働者とは自己の労働力を提供し、その対価としての賃金や給料によって生活する者。\cite{労働者}''
となる.
しかし,このレポートではもっと抽象的な話だと思うので,以下のように定義する.
\begin{tabular}{ll}
  国家  & 社会の秩序と統治を司り,法律や規制を通じて経済的・社会的なルールを定める存在. \\
  資本家 & 資本や資源を所有し,それを用いて経済活動を行い利益を追求する存在.       \\
  労働者 & 労働力を提供し,その対価として賃金を得る存在.                 \\
\end{tabular}

\section{本文}
国家・資本家・労働者の関係は経済と社会の基盤を形成する重要な要素であり,相互に依存しつつも対立する側面を持っている.
この関係性を理解するには,それぞれの役割と相互作用を考察することが不可欠である.

まず,国家は法律や規制を通じて社会秩序を維持し経済活動を調整する役割を果たす.
国家は,資本家の活動を支援しつつも労働者の権利を保護し,社会的な安定を保つ必要がある.
例えば,最低賃金法や労働条件の規制を設けることで,労働者が資本家によって不当な労働環境に置かれることを防ぐ役割を担っている.
また,国家はインフラ整備や公共サービスを提供し,資本主義経済が円滑に運営されるための環境を整える責任も負っている.

資本家は資本を所有し,それを使って商品やサービスを生産し利益を追求する.
彼らは労働者の労働力を利用して生産活動を行い,その結果得られる利益を再投資してさらに大きな資本を形成する.
この関係は資本家が生産活動を継続するために労働者の協力が不可欠であることを示しているが,資本家はしばしば利益の最大化を図るためにコスト削減や効率化を優先し労働条件を悪化させる傾向がある.

労働者は資本家によって提供される雇用機会を通じて労働力を提供し,その対価として賃金を得て生活を成り立たせている.
しかし,労働者の立場は資本家に対して弱いことが多く,労働者はしばしば労働条件の改善や賃金の引き上げを求めて闘わなければならない.
この対立関係は,労働運動やストライキを通じて資本家に圧力をかけ労働者の権利を主張するという形で現れる.

国家・資本家・労働者の関係は相互に依存しながらも対立するダイナミズムの中で成り立っている.
資本家は労働者を必要とし労働者は資本家の提供する雇用機会を必要としている.
一方で,国家はこの二者の間の対立を調整しバランスを取るための役割を果たしている.
現代においては,グローバル化や技術の進展により国家が資本家や労働者に与える影響も複雑化しているが,国家が公正なルールを設定し資本家と労働者の利益を調整する役割は依然として重要である.

今後,持続可能な社会を実現するためには国家が公正な規制を整備し,資本家と労働者の利益がよりバランスの取れた形で調整される必要がある.
国家がこの調整役として,資本家の利益だけでなく労働者の福祉にも配慮した政策を打ち出すことが求められている.
% 参考文献
\begin{thebibliography}{99}
  \bibitem{国家} コトバンク 国家とは\\
  \url{https://kotobank.jp/word/%E5%9B%BD%E5%AE%B6-65173}
  \bibitem{資本家} コトバンク 資本家とは\\
  \url{https://kotobank.jp/word/%E8%B3%87%E6%9C%AC%E5%AE%B6-523551#w-1172391}
  \bibitem{労働者} コトバンク 労働者とは\\
  \url{https://kotobank.jp/word/%E5%8A%B4%E5%83%8D%E8%80%85-662792#w-662792}
\end{thebibliography}

\end{document}