\documentclass[titlepage,a4paper]{jsarticle}
\usepackage{../../../sty/import}% 各種パッケージインポート
\usepackage{../../../sty/title}% タイトルページの変更

%% タイトルページの変数
% レポートタイトル
\title{社会福祉概論 第5回課題}
% 提出日
\expdate{\today}
% 科目名
\subject{社会福祉概論}
% 分野
\class{情報経営システム工学分野}
% 学年
\grade{B3}
% 学籍番号
\mynumber{24336488}
% 記述者
\author{本間三暉}
%
\begin{document}
% titleページ作成
\maketitle
社会福祉業界は,日本に高齢化社会での需要が高まっており,人材不足が問題となっています.
仕事内容には厳しい面もあるが,やりがいのある職場で社会貢献をしたいと考えている就活生からは注目を集めている業界である.
しかし,そのような就活生の中にも,仕事がハードであるとか給料が安く離職率が高いのではないかというマイナスなイメージは持っており,不安を感じている方もいる
ついては「福祉専門職」の業務や資格要件等をまとめなさい.
\section{社会福祉士の業務や資格所得要件}
社会福祉士は,福祉サービスを必要とする人々が抱える課題を解決するために適切な相談支援を行う専門職である.
具体的には,高齢者,障害者,子ども,生活困窮者など,さまざまな人々の相談に応じ,福祉サービスの利用支援や行政機関との連携を行う.

社会福祉士の資格取得要件は国家試験に合格することである.
国家試験を受験するためには指定された福祉系の大学や短期大学を卒業するか,養成施設を修了することが必要である.
また,実務経験がある場合には特例で受験資格が認められる場合もある.\cite{1}

\section{介護福祉師の業務や資格取得要件}
介護福祉士は,介護を必要とする人々に対し,身体介護や生活支援を提供する専門職である.日常生活のサポートだけでなく,利用者の心身の状態に応じた介護計画の作成や,他の医療福祉職との連携も重要な業務である.

介護福祉士の資格取得要件は,国家試験に合格することである.
受験資格を得るためには,養成校で必要なカリキュラムを修了する,介護職として一定の実務経験を積む,または特定の研修課程を修了するなどの要件を満たす必要がある.
また,近年では実務者研修が必須となっている.\cite{2}

\section{介護福祉師を増やすための方策}
介護福祉士を増やすためには,まず職場環境の改善が重要である.
長時間労働や低賃金といった課題に対処するため,介護報酬の引き上げや職員の負担軽減を図る施策が必要である.
また,介護職の魅力を広めるための広報活動も効果的である.例えば,若い世代に向けて介護福祉士のやりがいや社会的意義を訴えることで志望者の増加が期待できる.
さらに,資格取得に対する支援策として奨学金制度や教育費補助などを充実させることも有効である.\cite{3}
\begin{thebibliography}{99}
  \bibitem{1}社会福祉士とは | 日本社会福祉士会\url{https://www.jacsw.or.jp/citizens/cswtoha/}
  \bibitem{2}介護福祉士とは | 日本介護福祉士会\url{https://www.jaccw.or.jp/about/fukushishi}
  \bibitem{3}介護人材確保に向けた取組について|厚生労働省|厚生労働省\url{https://www.mhlw.go.jp/stf/newpage_02977.html}
\end{thebibliography}
\end{document}