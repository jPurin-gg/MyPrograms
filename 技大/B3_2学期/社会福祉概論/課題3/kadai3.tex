\documentclass[titlepage,a4paper]{jsarticle}
\usepackage{../../../sty/import}% 各種パッケージインポート
\usepackage{../../../sty/title}% タイトルページの変更

%% タイトルページの変数
% レポートタイトル
\title{社会福祉概論 第3回課題}
% 提出日
\expdate{\today}
% 科目名
\subject{社会福祉概論}
% 分野
\class{情報経営システム工学分野}
% 学年
\grade{B3}
% 学籍番号
\mynumber{24336488}
% 記述者
\author{本間三暉}
%
\begin{document}
% titleページ作成
\maketitle
\section{a.認知症を引き起こす主な病気と症状}
認知症は,脳の病気や障害によって認知機能が低下し日常生活に支障をきたす状態を指す.
主な原因となる病気には,アルツハイマー病,血管性認知症,レビー小体型認知症,前頭側頭型認知症がある.
アルツハイマー病は認知症の中で最も一般的であり,記憶障害や思考能力の低下が徐々に進行するのが特徴である\cite{3_a_1}.

血管性認知症は,脳の血管障害によって引き起こされ,脳梗塞や脳出血が原因となることが多い.
症状は認知機能の低下が段階的に現れることがあり,記憶障害に加えて,判断力や計画能力の低下もみられる.
レビー小体型認知症は,幻視や筋肉の硬直が特徴であり,進行に伴いパーキンソン症状が現れることもある.
前頭側頭型認知症は人格や行動の変化が初期に顕著であり,特に前頭葉の変性が進行することが特徴である\cite{3_a_2}.

\section{b.認知症サポーターの役割と期待されていること}
認知症サポーターは,認知症について理解を深め,認知症の人やその家族を温かく見守り,地域で支える役割を担う人々である.
認知症サポーターは,特別な介護技術を持つ必要はないが,認知症についての正しい知識を持ち,日常生活において困難を抱える人々への理解と配慮を示すことが期待されている.
また,地域社会における支援の一環として,認知症サポーターは周囲の人々に対して認知症に関する知識を広め,支援の輪を広げる役割も担っている\cite{3_b}.

このように,認知症サポーターは,地域社会全体で認知症の人々を支えるための要となる存在である.
彼らの活動によって,認知症の人々が住み慣れた地域で安心して暮らし続けることが可能となり,地域の一員としての生活が守られているのである.
% 参考文献
\begin{thebibliography}{99}
\bibitem{3_a_1}知っておきたい認知症の基本 | 政府広報オンライン\url{https://www.gov-online.go.jp/useful/article/201308/1.html}
\bibitem{3_a_2}家族の会の書籍 | 活動内容 | 公益社団法人認知症の人と家族の会 \url{https://www.alzheimer.or.jp/}
\bibitem{3_b}認知症サポーター |厚生労働省\url{https://www.mhlw.go.jp/stf/seisakunitsuite/bunya/0000089508.html}
\end{thebibliography}

\end{document}