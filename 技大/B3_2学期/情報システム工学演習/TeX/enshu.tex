\documentclass[titlepage,a4paper]{jsarticle}
\usepackage{../../../sty/import}% 各種パッケージインポート
\usepackage{../../../sty/title}% タイトルページの変更

%% タイトルページの変数
% レポートタイトル
\title{WebAPIを使用したスマートフォンアプリ}
% 提出日
\expdate{\today}
% 12/27 17:00まで
% 科目名
\subject{情報システム工学演習}
% 分野
\class{情報経営システム工学分野}
% 学年
\grade{B3}
% 学籍番号
\mynumber{24336488}
% 記述者
\author{本間三暉}

\begin{document}
% titleページ作成
\maketitle

\section{目的}
% このアンケートシステム(DataBase,Repository,Domain,Servlet,JSP+WebAPI+スマートフォンアプリ)の目的を記載すること。
本レポートの目的は,授業の一環として個人で開発したアンケートシステムについて,その企画・設計・実装の全過程を詳細に記録し,そこから得られた技術的知見や自己成長の成果を明確化することである.
本システムは,ユーザーが簡便かつ効率的にアンケートを作成・配布・集計できることを目指し,実用性と操作性を重視して設計された.
開発を通じて,Webアプリケーション開発の基礎知識やフレームワークの使用方法だけでなく,データベース設計やセキュリティ対策の重要性についても実践的に学習した.

本レポートでは,これらのプロジェクト全体を振り返り,成功した点や改善すべき点を整理し,次回の開発に生かすための具体的な指針を提示する.
特に,データの管理手法,ユーザーエクスペリエンスの向上,および個人開発における効率的な問題解決のアプローチについて考察する.
この振り返りを通じて,単なる技術習得にとどまらず,今後のより高度な開発への応用力を高めることを目指す.
\section{原理と構成}
% このシステムの構成を図で示し、どのような仕組みで動作するかを説明してください。
% (※図は手書きでもかまいません)

\section{動作確認結果}
% スマートフォン画面のキャプチャ画像などを使って、作成したシステムが正常に動作していることを示してください。

\section{考察}
%   以下について考察してください。
% ①    このシステムの長所
% ②    このシステムのSecurity
\subsection{このシステムの長所}


\subsection{このシステムのSecurity}
% 「’); delete from testAnswer; --」ではエラーが出ない.「<script>alert(document.cookie);</script>」ではアラートが出てしまう.cookieは存在するか怪しい.

\section{感想}
%  ※ここは採点の対象外です。
%   今回の演習の内容について感想などあれば記述して下さい。
%  次年度以降の演習の実施に役立てたいと思います。
for QandAと書いてあるのに中身がItemManagerなのをやめてほしい.
そもそも配らないか,しっかり中身を書き換えたものにしてほしい.

% 参考文献
\begin{thebibliography}{99}
  \bibitem{}
\end{thebibliography}

\end{document}