\documentclass[titlepage,a4paper]{jsarticle}
\usepackage{../../../sty/import}% 各種パッケージインポート
\usepackage{../../../sty/title}% タイトルページの変更
\renewcommand{\thesection}{課題\arabic{section}}
\setcounter{section}{18}
% 右上側に名前,学籍番号,提出日の記述が必要なので,急遽付け足し
\usepackage{fancyhdr}
\pagestyle{fancy}
  \lhead{11/3情報社会と著作権課題}
  \rhead{
    \begin{tabular}[t]{>{\bfseries}l@{\hspace{0.5\zw}:\hspace{0.5\zw}}l}
      氏名   & 本間三暉\\
      学籍番号 & 24336488\\
      提出日  & \today
    \end{tabular}
  }
%% タイトルページの変数
% レポートタイトル
\title{情報と著作権課題19〜21}
% 提出日
\expdate{\today}
% 科目名
\subject{情報社会と著作権}
% 分野
\class{情報経営システム工学分野}
% 学年
\grade{B3}
% 学籍番号
\mynumber{24336488}
% 記述者
\author{本間三暉}

\begin{document}
% titleページ作成
\maketitle
\section{ }
\begin{enumerate}
  \item 森のくまさん」替え歌事件について調べ,概要を説明してください.
  \item いわゆる替え歌事件について,
        \begin{itemize}
          \item 元の歌詞の一部を面白おかしく変更している場合,
          \item 元の歌詞が全く残っていない場合,
        \end{itemize}
        それぞれについて,同一性保持権の侵害になるかどうか,理由をつけて回答してください.
\end{enumerate}
\subsection{「森のくまさん」替え歌事件について調べ,概要を説明してください.}
「森のくまさん」替え歌事件は,2016年から2017年にかけて発生した著作権に関する問題である.
お笑い芸人のパーマ大佐氏が,童謡「森のくまさん」に独自の歌詞を追加した替え歌を制作し2016年12月にCDとして発売した.
しかし,この替え歌に対し,同曲の日本語訳詞者である馬場祥弘氏が,著作者人格権の侵害を主張し,CDの販売差し止めと慰謝料の支払いを求めた.\cite{bear}

馬場氏は,レコード会社から歌詞の改変許可を求められたがこれを拒否していた.それにもかかわらずCDが発売されたため,馬場氏は法的措置を取ることとなった.
最終的に,2017年2月1日パーマ大佐氏とレコード会社は馬場氏と合意に達し円満解決に至ったと報じられている.\cite{bear2}

この事件は著作権法における著作者人格権,特に同一性保持権の重要性を再認識させる事例となった.
また,パロディや替え歌の制作において,原作者の権利を尊重する必要性を示すものとなった.
\subsection{いわゆる替え歌事件について}
\subsection*{元の歌詞の一部を面白おかしく変更している場合}
元の歌詞の一部を残して替え歌を作成し面白おかしく改変する場合は,元の歌詞が改変されているため同一性保持権の侵害に該当する可能性が高い.
このような改変は著作者の意図を変えてしまう恐れがあり,元の歌詞に対する著作者の名誉や意図に反する形で使われることで侵害と見なされることがある.
改変部分が少ないとしても,原曲の雰囲気を損ねるような内容であれば,同一性保持権の侵害が成立しやすい.

\subsection*{元の歌詞が全く残っていない場合}
元の歌詞が全く残っていない場合,同一性保持権の侵害には当たらない可能性が高い.
なぜなら,元の歌詞を改変した内容が含まれていないため,著作者の意図や名誉に直接影響を与えることがないと考えられるためである.
この場合,元の著作物を参照したかどうかにかかわらず,同一性保持権の保護対象とはならないと考えられる.

\newpage
\section{「ミッキーマウス」の著作権が現在どのような状況かを調べてまとめてください.}
ミッキーマウスの著作権状況は作品の公開年や各国の著作権法により異なる.
1928年に公開された初登場作品『蒸気船ウィリー』に関して,アメリカでは著作権保護期間が95年であるため,
2023年12月31日に保護期間が終了し,2024年1月1日からパブリックドメインとなった.
これにより,アメリカ国内では同作品のミッキーマウスの二次使用が可能となっている.\cite{micky}

一方,日本におけるミッキーマウスの著作権保護期間は,解釈が複数存在し,確定していない.
日本の著作権法では,映画の著作物は公開から70年,美術の著作物は作者の死後70年と定められているが,
ミッキーマウスをどの著作物として扱うか,また戦時加算の適用などにより保護期間の終了時期に関して複数の見解がある.
そのため,現時点では日本においてミッキーマウスの著作権が切れていると断言することは難しい状況である.

さらに,著作権が切れたとしてもミッキーマウスには商標権や著作者人格権が存在するため,商用利用や改変には注意が必要である.
商標権は,商品やサービスのマークやネーミングなどを保護する権利であり,ディズニーは「ミッキーマウス」など多くの商標を登録している.
商標権の侵害は,差止めや損害賠償を請求される可能性があるため,著作物の使用には慎重な対応が求められる.

以上のように,ミッキーマウスの著作権状況は国や作品によって異なり,商標権や著作者人格権など他の権利も関係するため,利用に際しては最新の情報を確認し適切な対応を取ることが重要である.

\section{ゲスト講師の講演について感想を簡潔に述べてください.}
講演では,「著作物の利用」や「著作権の活用」について,森永製菓の事例を通じて解説がありました.
ゲスト講師の公演を通じて著作権が企業戦略や製品開発にどのように活用されているかがよくわかりました.

\newpage
% 参考文献
\begin{thebibliography}{99}
  \bibitem{bear}「森のくまさん」替え歌ヒドすぎ、パーマ大佐を訴え - 芸能 : 日刊スポーツ\url{https://www.nikkansports.com/entertainment/news/1767126.html}
  \bibitem{bear2}「森のくまさん」騒動が“円満解決” 訳詞者・馬場氏に誠意伝わる | ORICON NEWS\url{https://www.oricon.co.jp/news/2085623/full/}
  \bibitem{micky}ミッキーマウスの著作権は切れた?日本での保護期間や自由に使えるのかを解説 | IP mag - IPの可能性を広げるエンタメ経済メディア\url{https://ipmag.skettt.com/detail/mickey-mouse-copyright}
\end{thebibliography}

\end{document}