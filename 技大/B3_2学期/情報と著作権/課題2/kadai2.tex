\documentclass[titlepage,a4paper]{jsarticle}
\usepackage{../../../sty/import}% 各種パッケージインポート
\usepackage{../../../sty/title}% タイトルページの変更
\renewcommand{\thesection}{課題\arabic{section}}
\setcounter{section}{1}
%% タイトルページの変数
% レポートタイトル
\title{情報と著作権課題2〜6}
% 提出日
\expdate{\today}
% 科目名
\subject{情報と著作権}
% 分野
\class{情報経営システム工学分野}
% 学年
\grade{B3}
% 学籍番号
\mynumber{24336488}
% 記述者
\author{本間三暉}

\begin{document}
% titleページ作成
\maketitle
\section{著作物に「©」というマークが付けられていることがあります.この©マークについて調べてください.
  うまく見つけられない人は「著作権マーク」などで検索してみてください.}
©マーク(著作権マーク)は,著作権の存在を示す記号であり,著作物が著作権法で保護されていることを知らせる役割を果たしている.
日本はベルヌ条約に加盟しており,著作権は著作物の創作時に自動的に発生する「無方式主義」を採用しているため,法的には©マークの表記は必須ではない.
ただし,著作権者や著作物の発行年を明示するため,©マークを使用することには意義がある\cite{utility}\cite{authense}.

©マークを表示することで第三者に著作物が保護されていることを周知し,無断転載やコピーを抑止する効果も期待できる\cite{ameni}.
この記号には,著作権者名や発行年とともに「© 著作権者名 発行年」といった形式で記載することが推奨されている\cite{canvas}.
また,著作物が他国で利用される際にも保護を示す手段となる\cite{tarunk}.
\section{最近(数年以内くらい)で,著作権法違反により容疑者が逮捕された事件を挙げて,その事件の概要を説明してください.
  なお,デジタル化,ネットワーク化に関係のあることにしてください.}
2022年11月に静岡県の会社員男性が,著作権法違反の容疑で逮捕された\cite{ACCS}.
この男性は,人気ゲームソフト「信長の野望」や「スーパーマリオアドバンス」などを無許可で複製したデータをSDカードに入れ,
模倣ゲーム機とともにインターネットオークションで販売していた.
宮城県警のサイバーパトロールによって発見され,著作権者への通報を経て捜査が行われた.
デジタル化とネットワーク化の進展に伴い,新たな形で著作権侵害が発生していることを示す事例である.
\section{著作権法において,無方式主義が採用されている理由を検討し,回答してください.}
著作権法で無方式主義が採用されている理由は,創作と同時に自動的に著作権保護が付与されるため登録などの手続きが不要で創作者の権利を迅速に守れる点にある.
形式的な登録の負担をなくし,創作活動を促進する意図がある.
さらに,ベルヌ条約のような国際条約が無方式主義を採用しているため,国際的な著作権保護の一貫性も確保できる.
\section{著作権以外の知的財産権として産業財産権があります.産業財産権の種類とそれぞれの概要を回答してください.}
産業財産権は特許権,実用新案権,意匠権,商標権の4種類があり,それぞれ異なる対象と範囲で知的財産を保護する.
これらの権利は,創作者や企業の知的財産を保護し公平な競争を促進することで産業の発展に寄与するものである.
\subsection{特許権}
特許権は新しい発明に対して与えられる権利で,技術的に高度なアイデアや解決方法が対象である.
この権利を得るためには産業上の利用可能性,新規性,進歩性が要求され,特許庁の厳格な審査を通過する必要がある\cite{5_1}.
\subsection{実用新案権}
実用新案権は既存のアイデアに改良を加えて使い勝手を向上させたような小発明に対して与えられる権利である.
特許権と異なり無審査で登録されるが,権利行使には「実用新案技術評価書」の提示が必要とされ,保護範囲は比較的限定的である\cite{5_2}.
\subsection{意匠権}
意匠権は製品の独自性あるデザインを保護するための権利である.
これは物品の形状や模様,色彩などの外観に関するデザインに適用される.
例えば,立体的な製品デザインや視覚的に特徴的な構造に対して取得可能で,その美的価値を守る役割を果たす\cite{5_3}.
\subsection{商標権}
商標権は企業や製品を識別するためのマークやロゴ,名称などに与えられる権利である.
ブランドイメージを保護するための権利であり,更新手続きを行うことで半永久的に維持可能である.
これにより,長期にわたり消費者の信頼を保つことができる\cite{5_4}.
\section{以下の著作物について,それぞれ具体例を挙げて説明してください}
\subsection{言語の著作物}
\subsubsection*{村上春樹の『ノルウェイの森』}
言語の著作物は,小説や詩など,文字や言葉で表現された作品である.
この作品では,文字による文章表現を通じて,登場人物の感情や情景が読者に伝わる.
例えば,村上春樹の『ノルウェイの森』は,文章を通して主人公の心の動きや人間関係が描かれているため,典型的な言語の著作物といえる.
\subsection{音楽の著作物}
\subsubsection*{ベートーヴェンの交響曲第5番}
音楽の著作物は,メロディやリズム,ハーモニーを通して表現され,楽譜に記録されることで具体化する.
この交響曲第5番は「運命」とも呼ばれ,印象的なメロディが特徴で聴衆に強い感情やドラマを感じさせる.
演奏を通じて再現される音楽作品は,音の構造そのものが創作物である.
\subsection{舞踊,無言劇の著作物}
\subsubsection*{バレエ『白鳥の湖』}
舞踊や無言劇の著作物は身体の動きや表情によって感情や物語が伝わるもので,台詞を用いずに視覚的に鑑賞される.
『白鳥の湖』では,バレエダンサーが繊細な振り付けや演技を通じて登場人物の感情や物語の展開を表現しており,観客に感動を与える.
\subsection{美術の著作物}
\subsubsection*{ピカソの『ゲルニカ』}
美術の著作物は,絵画や彫刻など視覚的に鑑賞される作品である.
ピカソの『ゲルニカ』は独自の構図と色彩によって戦争の悲惨さを表現しており,強いメッセージを伝える.
形や色彩の創意が著作物として評価されるため,美術作品は視覚的な創作物として認められる.
% 参考文献
\begin{thebibliography}{99}
  \bibitem{utility}コピーライト(Copyright ©)の意味と書き方とは?記号の使い方や表記方法・3つの必須項目を解説 \url{https://utilly.jp/article/copyright/}
  \bibitem{authense}著作権マークの表記は必須?付け方・使い方を弁護士がわかりやすく解説 | Authense法律事務所\url{https://www.authense.jp/komon/blog/ip/2803/}
  \bibitem{ameni}【著作権表示】「all rights reserved」や「©」の正しい意味や使い方全部解説!|AMEMI\url{https://amemi.jp/archives/3251}
  \bibitem{canvas}コピーライトの正しい書き方をマスター! 「©」表記の意味や必要性も解説! | 株式会社キャンバス |Web制作やSEOのノウハウをお届けします\url{https://www.canvas-works.jp/journal/copyright/}
  \bibitem{tarunk}コピーライトの意味とは。all rights reservedは必要?正しい表記や書き方を徹底解説 | MarkeTRUNK\url{https://www.profuture.co.jp/mk/column/about-copyright}
  \bibitem{ACCS}著作権侵害事件 | ACCS\url{https://www2.accsjp.or.jp/criminal/2022/1141.php}
  \bibitem{5_1}知財管理の大きなポイントになる産業財産権とは | 弁理士法人 平和国際特許事務所\url{https://www.heiwa-pat.com/column/theme01/column03.php}
  \bibitem{5_2}産業財産権の種類 | とうほく知的財産いいねっと\url{https://www.tohoku.meti.go.jp/chizai-enet/about_chizai/chizai/about_sangyo.html}
  \bibitem{5_3}産業財産権(特許権、実用新案権、意匠権、商標権)とは? | カブト弁理士の転職相談ブログ\url{https://kabuto-benrishi.com/industrial-property-rights/}
  \bibitem{5_4}4種類の産業財産権(特許権、実用新案権、意匠権、商標権)の違いについて | 知財辞苑\url{https://tizai-jien.co.jp/2017/08/23/post_55/}
\end{thebibliography}

\end{document}