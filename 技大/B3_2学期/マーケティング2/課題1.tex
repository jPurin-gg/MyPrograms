\documentclass[titlepage,a4paper]{jsarticle}
\usepackage{../../sty/import}% 各種パッケージインポート
\usepackage{../../sty/title}% タイトルページの変更
\usepackage{otf}% ローマ数字を使えるようにするやつ
\renewcommand{\thesection}{課題\arabic{section}}
%% タイトルページの変数
% レポートタイトル
\title{マーケティング\ajRoman{2}課題まとめ}
% 提出日
\expdate{\today}
% 科目名
\subject{マーケティング\ajRoman{2}}
% 分野
\class{情報経営システム工学分野}
% 学年
\grade{B3}
% 学籍番号
\mynumber{24336488}
% 記述者
\author{本間三暉}
% グループ名 % もし班があるやつならtitle_team.styを入れる
% \team{10}
% 共同実験者 % もし共同実験者が必要なやつならtitle_kyoudou.styを入れる
% \coauthor{%
% \textbf{学籍番号:} & \textbf{氏名:} \\
% \textbf{学籍番号:} & \textbf{氏名:}\\
% \textbf{学籍番号:} & \textbf{氏名:}\\
% \textbf{学籍番号:} & \textbf{氏名:}\\
% }
%
% 記載例:
%\coauthor{%
% 学籍番号:24567321 & 氏名:吉田 富美男 \
% 学籍番号:12345678 & 氏名:安藤 雅洋 \
% 学籍番号:13579234 & 氏名:雲居 玄道 \
%%

\begin{document}
% titleページ作成
\maketitle
\section{スイッチングコストが高いと考える商品、またはサービスをひとつ選び、その概要について説明してください。}

\section{独自のポジションを獲得できていると考える商品、または企業をひとつ選び、その特徴を示すポジショニングマップを 作成するときの分析軸(2軸)を選び、対象の商品(企業)の位置づけなどについて説明する。}

\section{}
% 参考文献
\begin{thebibliography}{99}
  \bibitem{slide1} 令和6年 マーケティング\ajRoman{2}授業スライド
  \bibitem{}
\end{thebibliography}

\end{document}