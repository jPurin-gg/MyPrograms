\documentclass[titlepage,a4paper]{jsarticle}
\usepackage{../../sty/import}% 各種パッケージインポート
\usepackage{../../sty/title}% タイトルページの変更
\renewcommand{\thesubsection}{\thesection-\arabic{subsection})}
%% タイトルページの変数
% レポートタイトル
\title{インタラクティブ・システム・デザイン期末レポート(課題1)}
% 提出日
\expdate{\today}
% 科目名
\subject{インタラクティブ・システム・デザイン}
% 分野
\class{情報経営システム工学分野}
% 学年
\grade{B3}
% 学籍番号
\mynumber{24336488}
% 記述者
\author{本間三暉}

\begin{document}
% titleページ作成
\maketitle

\section{デザイン解の形式として可能なものを,3個,あげなさい(各20文字以内)}
\subsection{ }
\begin{itemize}
  \item メニュー選択形式(8文字)
  \item 音声入力形式(6文字)
  \item ジェスチャ操作形式(9文字)
\end{itemize}

\subsection{ }
\begin{itemize}
  \item タスク分析に基づく選択(12文字)
  \item ユーザビリティ重視の選択(14文字)
  \item 環境適応型選択(9文字)
\end{itemize}

\subsection{ }
\begin{itemize}
  \item プロトタイプ評価(9文字)
  \item ユーザニーズ調査(9文字)
  \item タスク遂行観察(8文字)
\end{itemize}

\section{そして,どのように選択を行うのか,また,その選択を行うのに問題定義の他の側面(「ユーザ」と「支援のレベル」)が寄与するのかについて検討しなさい}
\renewcommand{\thesubsection}{\thesection)}
\subsection{ }
デザイン解の選択はユーザビリティとタスクの性質に基づき行う.
例えば「ジェスチャ操作形式」は直感的だが,操作環境が制約される場合に不適となる.
ここで問題定義の側面「ユーザ」と「支援のレベル」が寄与する.
タスク遂行中のユーザ特性や要求される支援のレベル(例:迅速性や正確性)がデザイン形式を選ぶ基準となる.
また,インタラクティブシステムの評価を通じ,最適な解を導出するプロセスが必要である.(193文字)
% 参考文献
\begin{thebibliography}{99}
  \bibitem{index}インタラクティブ・システム・デザイン資料
\end{thebibliography}

\end{document}