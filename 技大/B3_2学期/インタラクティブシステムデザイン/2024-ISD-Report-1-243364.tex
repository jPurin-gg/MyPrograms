\documentclass[titlepage,a4paper]{jsarticle}
\usepackage{../../sty/import}% 各種パッケージインポート
\usepackage{../../sty/title}% タイトルページの変更
\renewcommand{\thesubsection}{\thesection-\arabic{subsection})}
%% タイトルページの変数
% レポートタイトル
\title{インタラクティブ・システム・デザイン期末レポート(課題1)}
% 提出日
\expdate{\today}
% 科目名
\subject{インタラクティブ・システム・デザイン}
% 分野
\class{情報経営システム工学分野}
% 学年
\grade{B3}
% 学籍番号
\mynumber{24336488}
% 記述者
\author{本間三暉}

\begin{document}
% titleページ作成
\maketitle
\begin{center}
  インタラクティブ・システム・デザイン期末レポート(課題1)
\end{center}
\begin{flushright}
  \today

  24336488,本間三暉
  \vskip\baselineskip % 一行空行
\end{flushright}
\section{デザイン解の形式として可能なものを,3個,あげなさい(各20文字以内)}
\subsection{ }
ポップアップ辞書(8文字)
\subsection{ }
リアルタイム翻訳(8文字)
\subsection{ }
単語マーカー支援(8文字)
\section{そして,どのように選択を行うのか,また,その選択を行うのに問題定義の他の側面(「ユーザ」と「支援のレベル」)が寄与するのかについて検討しなさい}
\renewcommand{\thesubsection}{\thesection)}
\subsection{ }
選択は「ユーザの使いやすさ」と「支援レベル」を基準に行う.
ポップアップ辞書はシンプルで初心者向け,リアルタイム翻訳は高度な支援を提供し,単語マーカー支援は中級者の学習向けである.
「ユーザ」は文書を読む学習者や翻訳者であり,目的やスキルレベルに応じて最適な支援形式が異なる.
「支援のレベル」は,タスク効率向上や認知的負担の軽減を基準とする.
例えば,高度な支援ほど速読を可能にするが,ユーザの自主学習を妨げる可能性もある.
したがって,ユーザの習熟度と必要な支援レベルを明確にすることで,適切なデザイン解が選択できる.(256文字)
% 参考文献
\begin{thebibliography}{99}
  \bibitem{index}インタラクティブ・システム・デザイン資料
\end{thebibliography}

\end{document}