\documentclass[titlepage,a4paper]{jsarticle}
\usepackage{../sty/import}% 各種パッケージインポート
\usepackage{../sty/title}% タイトルページの変更

%% タイトルページの変数
% レポートタイトル
\title{政治とは何か}
% 提出日
\expdate{\today}
% 科目名
\subject{政治学}
% 分野
\class{情報経営システム工学分野}
% 学年
\grade{B3}
% 学籍番号
\mynumber{24336488}
% 記述者
\author{本間三暉}
% グループ名 % もし班があるやつならtitle_team.styを入れる
% \team{10}
% 共同実験者 % もし共同実験者が必要なやつならtitle_kyoudou.styを入れる
% \coauthor
% 記載例:
%\coauthor{%
% 学籍番号:24567321 & 氏名:吉田 富美男 \
% 学籍番号:12345678 & 氏名:安藤 雅洋 \
% 学籍番号:13579234 & 氏名:雲居 玄道 \
%%
%%%%%%%%%%%%%%%%%
%% 100%ChatGPT
%%%%%%%%%%%%%%%%%
\begin{document}
% titleページ作成
\maketitle
\section{はじめに}
政治とは,社会における権力と権威の分配,管理,使用を指す概念であり,個人や集団の行動や意思決定に大きな影響を与えます.
本レポートでは,映画「チョコレート・ドーナツ」と「パンズ・ラビリンス」を題材に,それぞれの作品に描かれた政治的なテーマを比較し,政治とは何かについて論じます,.
両作品に共通する政治的なメッセージを探り,政治の本質を明らかにすることを目的とします.
また,過去の授業で扱った「ロード・オブ・ザ・リング」「ノー・マンズ・ランド」「12人の怒れる男」「クリムゾン・タイド」との比較も交えて考察を深めます.
\section{映画の概要}
\subsection{チョコレート・ドーナツ}
「チョコレート・ドーナツ」は,1970年代のアメリカを舞台に,同性愛カップルであるルディとポールが,ダウン症の少年マルコを養子に迎えるために法的闘争を繰り広げる物語です.
ルディ役をアラン・カミング,ポール役をギャレット・ディラハントが演じ,社会の偏見と戦う彼らの姿を描いています.
\subsection{パンズ・ラビリンス}
「パンズ・ラビリンス」は,1944年のフランコ政権下のスペインを舞台に,少女オフェリアがファンタジーの世界に入り込み,現実の残酷さと戦いながら成長する物語です.
監督はギレルモ・デル・トロ,オフェリア役をイバナ・バケロが演じ,独裁政治の恐怖と暴力を背景にしたファンタジードラマです.
\section{政治的テーマの比較}
\subsection{権力と支配}
「チョコレート・ドーナツ」では,法律や社会制度の中での権力の使われ方が描かれ,同性愛カップルに対する偏見と差別が強調されます.
法の下での平等が求められる中,彼らが直面する障害は,政治的な権力構造の不公平さを浮き彫りにしています.

一方,「パンズ・ラビリンス」では,フランコ政権の抑圧的な権力が描かれ,独裁政治の恐怖と暴力が中心テーマとなっています.
ビダル大尉の冷酷な支配は,権力の濫用とそれに対する人々の抵抗を象徴しています.
\subsection{闘争と統合}
「チョコレート・ドーナツ」では,家族を守るための法的闘争が描かれています.
ルディとポールの闘いは,愛と絆がもたらす統合の力を示しています.
彼らの目的は,子供が幸せになることであり,法的勝利がその手段であることが強調されます.

「パンズ・ラビリンス」では,独裁政権に対する抵抗運動が描かれています.
オフェリアがファンタジー世界での闘争を通じて成長する様子は,現実世界での統合の試みと対比されています.
彼女の幻想的な冒険は,自由と安全を求める人間の本質を象徴しています.
\section{映画から見える政治の本質}
\subsection{人間の尊厳と権利}
「チョコレート・ドーナツ」は,マイノリティの権利と尊厳の重要性を強調しています.
同性愛カップルが直面する差別や偏見は,現代社会における人権問題を鋭く問いかけます.
彼らの闘いは,人間の尊厳を守るための政治的闘争の一例です.

「パンズ・ラビリンス」は,個人の自由と独裁政権の抑圧との対立を描いています.
オフェリアの冒険は,自由を求める人間の本能と,その自由を奪おうとする権力との間の葛藤を象徴しています.

\subsection{目的と手段}
「チョコレート・ドーナツ」では,子供の幸福を守るための闘いが,いつしか法的な勝利を求める闘争に変わる様子が描かれています.
これは,政治的な闘争が時として目的を見失い,手段そのものが目的化する危険性を示唆しています.

「パンズ・ラビリンス」では,自由と安全を求めるための幻想と現実の闘いが描かれています.
オフェリアの冒険は,目的を達成するために必要な手段の選択が,時に人間の本質を試すことを示しています.

\subsection{過去の授業で扱った映画との比較}
% 「ロード・オブ・ザ・リング」では,サウロンの圧倒的な権力とその支配に対抗するために多様な種族が団結する様子が描かれています.
% この映画は,権力の腐敗とその抑制の必要性を強調しています.

% 「ノー・マンズ・ランド」では,バルカン戦争中の敵対する兵士たちが同じ塹壕に閉じ込められる状況を通じて,戦争の無意味さと人間の尊厳について問いかけています.
% これは,「チョコレート・ドーナツ」で描かれる社会の不平等と同様に,個人の尊厳が政治の中心であるべきことを示しています.

% 「12人の怒れる男」は,陪審員たちが一人の少年の運命を決定する際の議論を通じて,司法制度の公正さとその脆弱性を描いています.
% この映画は,「チョコレート・ドーナツ」の法的闘争と通じるテーマを持ち,法の下での平等と正義の重要性を強調しています.

% 「クリムゾン・タイド」では,潜水艦の艦長と副艦長が核ミサイル発射をめぐって対立する様子を描き,権威と命令の服従,倫理的判断の重要性を問いかけています.
% この映画のテーマは,「パンズ・ラビリンス」での権力の濫用とその抑制の必要性と通じています.
「ロード・オブ・ザ・リング」におけるサウロンの権力は,「パンズ・ラビリンス」のビダル大尉の権力と類似しており,どちらも絶対的な支配を象徴しています.
このような絶対的権力に対抗するためには,多様なグループや個人が協力し合う必要があるというメッセージは,政治における統合の重要性を強調しています.

「ノー・マンズ・ランド」の兵士たちの境遇は,「チョコレート・ドーナツ」のルディとポールの闘争と同じく,個々の人間の尊厳を守るための闘いを描いています.
政治的な対立や戦争の中でも,人間の尊厳を守ることが最も重要であるというメッセージは,どちらの映画にも共通しています.

「12人の怒れる男」における陪審員たちの議論は,民主主義のプロセスとその複雑さを描いています.
公正な裁判を受ける権利は,「チョコレート・ドーナツ」における法的闘争と共通するテーマであり,法の下での平等と正義が政治の根幹であることを示しています.

「クリムゾン・タイド」の艦長と副艦長の対立は,命令と服従,そして倫理的判断の必要性を強調しています.
これは,「パンズ・ラビリンス」における権力の濫用とそれに対する抵抗のテーマと共通しています.
\section{授業内容との関連}
これまでの授業で学んだ政治の理論や概念を映画に当てはめて考察します.
非暴力的手段とポジティブな意見の方向性に関する議論や,政治における闘争と統合のバランスについての考察を踏まえ,両作品における政治的テーマを分析します.

授業では,政治とは100\%の闘争や統合があり得ないこと,闘争と統合がブレンドしている状態が政治の本質であると学びました.
この視点から見ると,「チョコレート・ドーナツ」と「パンズ・ラビリンス」は,いずれも政治の本質を描いていると言えます.

\section{まとめ}
「チョコレート・ドーナツ」と「パンズ・ラビリンス」は,一見すると政治的な題材としては異なる性質を持つ作品ですが,共に政治の本質を鋭く描いています.
両作品に共通するのは,権力と支配,闘争と統合のテーマであり,これらを通じて人間の尊厳や権利,自由の追求とその困難さが浮き彫りにされています.

また,「ロード・オブ・ザ・リング」,「ノー・マンズ・ランド」,「12人の怒れる男」,「クリムゾン・タイド」との比較を通じて,
これらの映画が描く政治の複雑さとその多面的な側面をより深く理解することができます.
政治とは,単なる権力の行使ではなく,人間の尊厳を守り,正義を追求するための闘いであり,その過程で生じる統合と分裂の連続であると言えます.

これらの映画は,政治の複雑さとその重要性を視覚的に描き出すことで,私たちに深い洞察を与えてくれます.
\end{document}