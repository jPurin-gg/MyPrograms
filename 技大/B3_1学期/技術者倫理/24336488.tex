\documentclass[titlepage,a4paper]{jsarticle}
\usepackage{../sty/import}%各種パッケージインポート
\usepackage{../sty/title}%タイトルページの変更
% レポートタイトル
\title{自分の会社の不正を告発する}
% 提出日
\expdate{\today}                      
% 科目名
\subject{技術者倫理}          
% 分野  
\class{情報経営システム工学分野}          
% 学年 
\grade{B3}                
% 学籍番号             
\mynumber{24336488}
% 記述者
\author{本間三暉}
% グループ名
% \team{10}
\begin{document}
\maketitle
\section{はじめに}
会社の不正を告発することは,倫理的な義務と同時に大きなリスクを伴う行動である.
不正を見逃すことは会社の存続に悪影響を及ぼし,社会的責任を果たせないことになる.
しかし,不正を告発することにより,自分自身のキャリアや生活に悪影響を及ぼす可能性もある.
ここでは,告発を行う際の手順と戦略について述べる.
\section{自分のタイプとその理由}
私は凡人であると自認している.その理由は,自分の能力や知識は平均的であり,特に突出した才能や能力を持っていないからである.
しかし,努力と協力を重んじる姿勢から,チームとしての活動において貢献できると考えている.
\section{選ぶべき仲間とその理由}
仲間を選ぶとしたら,秀才と天才の二人を選ぶ.秀才は,知識と技術に長けており,論理的かつ実務的なアプローチで問題を解決する能力がある.
彼は不正を告発するための具体的な証拠を集めるのに適している.
一方,天才は創造的な発想と卓越した分析力を持っており,問題の根本原因を見抜く力がある.
彼は告発の計画を立案し,成功に導くための戦略を提供することができる.
\subsection{チームの役割}
チームの役割を以下に示す.
\begin{itemize}
  \item 自分(凡人): 証拠の収集と整理を担当する.秀才や天才が見つけた情報を基に,具体的な証拠を集め,整理する役割を果たす.
  \item 秀才: 証拠の分析と法的な準備を担当する.法律に詳しく,適切な手続きを踏むための助言を提供する.
  \item 天才: 告発の戦略とプレゼンテーションを担当する.創造的なアプローチで問題解決に導き,最も効果的な方法で告発を行う計画を立てる.
\end{itemize}
\section{自分の不利益を最小化する作戦}
\begin{itemize}
  \item 匿名性の確保: 可能な限り匿名で告発を行う.内部告発用のホットラインや第三者機関を利用する.
  \item 証拠の確保: 告発に必要な証拠を十分に集める.信頼性のある証拠を持つことで,告発の正当性を示す.
  \item 法律の理解: 労働法や内部告発に関する法律を理解し,自分の権利を守る.
  \item 支援を得る: 信頼できる同僚や弁護士の支援を得る.個人の力だけではなく,チームで行動することが重要.
  \item リスク評価: 告発のリスクと利益を評価し,最悪のシナリオに備える.事前に退職を視野に入れた準備を行うことも考慮する.
\end{itemize}
\section{結論}
会社の不正を告発することは簡単な決断ではないが,倫理的な責任を果たすためには重要な行動である.
凡人である私が選んだ仲間の秀才と天才と共に,戦略的に告発を行い,自分自身の不利益を最小限に抑える方法を取ることで,社会正義を守ることができると信じている.
\end{document}