\documentclass[titlepage,a4paper]{jsarticle}
\usepackage{../sty/import}% 各種パッケージインポート
\usepackage{../sty/title}% タイトルページの変更
%% セクションの変更
% 1.→課題1とするもの
\renewcommand{\thesection}{課題\arabic{section}}
%% タイトルページの変数
% レポートタイトル
\title{ヒューマンインターフェース\\最終レポート}
% 提出日
\expdate{\today}
% 科目名
\subject{ヒューマンインターフェース}
% 分野
\class{情報経営システム工学分野}
% 学年
\grade{B3}
% 学籍番号
\mynumber{24336488}
% 記述者
\author{本間三暉}

\begin{document}
% titleページ作成
\maketitle
\section{現代・近未来のヒューマンインタフェースについて各自で調べ、A4用紙1枚以内にまとめなさい。}%1
%%%
\subsection*{仮想現実と現実拡張について}
仮想現実と現実拡張について
仮想現実(VR: Virtual Reality)とは\cite{VR},コンピュータによって作り出された仮想的な空間などを現実であるかのように疑似体験できる仕組みである.
VRは,ヘッドマウントディスプレイのように顔に装着するディスプレイとして実現される.
これによって,立体的な映像を出力する.
さらに顔の向きや傾き,さらにはユーザの位置に合わせて映像を制御することにより,ユーザが仮想空間に入り込んだかのような感覚を得ることができる.

拡張現実(AR: Augmented Reality)とは\cite{AR},仮想空間の情報やコンテンツを現実世界に重ね合わせて表示することによって,現実を拡張する技術や仕組みを指す.
日常では,スマートフォンのカメラで特定のマーカーを移すことによって,映像を投影したり,現実世界の映像に仮想世界の映像を合成して映し出すことで利用される.
そのほかにも,スマートフォンのアプリで,現実の山を映すとその山の名前が表示できるアプリ,家具を部屋に置いたときの状況を確認できるアプリなどが登場している.
ARグラスやスマートグラスなどと呼ばれる,メガネ型のデバイスも登場している.
これはレンズ部分にコンテンツを表示してARを実現するデバイスである.
ARとMRではアプローチは少し異なるものの,現実世界と仮想世界を結びつけるという点においては同じような技術と考えることができる.
また,技術的な進歩により,実現方法も進化して来ている.

近年では,ARとVRを融合させ,さらに発展させた複合現実(MR: Mixed Reality)\cite{MR}といった技術が進歩している.
頭部に装着したデバイスがユーザの周囲の空間や手などを認識し,仮想世界と現実世界のオブジェクトを立体感を考慮した映像合成が実現できる.
また,ユーザがデジタル情報を触って操作できることや,複数人で同時作業といった体験を可能にする.

Appleは``Vision Pro''というVR/ARデバイスを発売した.
Appleはこの製品を空間コンピュータと呼んでおり,頭部に身に着け,生活に溶け込むコンピュータという,これまでのMRデバイスの先を行くビジョンが感じられる.
このデバイスは身に着けている人の視線,音声,ジェスチャーを組み合わせてコンピュータを操作できる.
さらに,周囲の人の声や存在なども認識し,自然な表示・演出が可能である.

従来のコンピュータとAR,VRでは,Windowsを介して情報の入出力を行ったり,平面空間上に3次元オブジェクトを表示したりすることが一般的だった.
今後は,現実世界の空間とCGを融合させたアプリケーションが設計できると期待されている.
\section{感覚の種類と構造について、文章中の(1)〜(15)に適当な単語をいれよ。提出用書類には番号と答えのみ記載すること。}%2
\begin{enumerate}
      \item 特殊感覚
      \item 深部感覚
      \item 体性感覚
      \item 角膜
      \item 瞳孔
      \item 水晶体
      \item 網膜
      \item 桿体
      \item 錐体
      \item 蝸牛
      \item 平衡
      \item 匂い
      \item 味蕾
      \item 温度感覚
      \item 血流量
\end{enumerate}

\section{つぎの問いに答えよ。提出用書類には番号と答えのみ記載すること。}%3
\begin{enumerate}
      \subsection{明るさの恒常性の実験A(左下図)で、観察者が選ぶ比較刺激は、標準刺激に比べて濃い・同等・薄いのうちどれか?またその理由も述べよ。}
      \item 比較刺激は標準刺激と同等となる. \\
            感覚器官は,明るさを判断する際に,対象物とその背景の輝度の比率で刺激の強さをとらえていると考えられるからである.
            輝度とは客観的な明るさであり,物体の表面からの反射光の強さ,つまり,照明光の強さ×物体の反射率で表される.
            輝度の大きさで判断している場合標準刺激と 比較刺激が同じ輝度になるときは,ライトで照らされている側が照明光の強さが大 きいため,反射率が小さい濃い色となるはずである.
            しかし,実験結果では標準刺 激と比較刺激は同じ程度の濃度となることから,感覚器官は,背景と対象物の輝度 の比率で明るさを判断していると考えられる.
            \subsection{明るさの恒常性の実験B(右下図)で、観察者が選ぶ比較刺激は、標準刺激に比べて濃い・同等・薄いのうちどれか?またその理由も述べよ。}
      \item 比較刺激は標準刺激より濃くなる. \\
            観察者は小穴を通して,標準刺激と比較刺激の明るさを観察するようになるため,それぞれの対象物の表面からの輝度のみで判断することになる.
            よって,標準刺激と比較刺激が同じ輝度になるときは,ライトで照らされている側の比較刺激の反射率が小さいときで,濃度が濃いときとなる.
\end{enumerate}
\section{学習について、文章中の(1)〜(16)に適当な単語をいれよ。提出用書類には番号と答えのみ記載すること。}%4
\begin{enumerate}
      \item 行動
      \item 馴化
      \item 自発的回復
      \item 脱馴化
      \item 初期学習
      \item コンラート・ローレンツ
      \item 刷り込みの臨界期
      \item 古典的条件付け
      \item 無条件刺激
      \item 無条件反応
      \item 中性刺激
      \item 条件刺激
      \item 条件反応
      \item オペラント条件づけ
      \item 報酬
      \item 嫌悪刺激
\end{enumerate}
\section{文章中の(1)〜(10)に適当な単語をいれよ。同じ番号には同じ単語が入る。提出用書類には番号と答えのみ記載すること。}%5
\begin{enumerate}
      \item 末梢
      \item 中枢
      \item 脳
      \item 脊髄
      \item 体性
      \item 自律
      \item 知覚
      \item 運動
      \item 交感
      \item 副交感
\end{enumerate}
\section{文章中の(1)〜(19)に適当な単語をいれよ。同じ番号には同じ単語が入る。提出用書類には番号と答えのみ記載すること。}%6
\begin{enumerate}
      \item 細胞体
      \item 樹状突起
      \item 軸索
      \item 活動電位
      \item シナプス小胞
      \item 大脳皮質
      \item 脳溝
      \item 脳回
      \item 脳梁
      \item 一次運動野
      \item 体性感覚野
      \item 一次感覚野
      \item 連合野
      \item ブローカ野
      \item ウェルニッケ野
      \item 大脳辺縁系
      \item 視床
      \item 視床下部
      \item 恒常性
\end{enumerate}
\section{つぎの問いに答えよ。提出用書類には番号と答えのみ記載すること。}%7
\begin{enumerate}
      \subsection{脳波のアルファベットの略字は?}
      \item EEG
            \subsection{脳波計測の際、国際的に決められている電極の設置場所は何法という?}
      \item 10-20電極法
            \subsection{脳波計測に使われる電極は探査電極ともうひとつは?}
      \item 基準電極
            \subsection{問題3のふたつの電極を組み合わせた計測法をなんという?}
      \item 基準導出法
            \subsection{レム睡眠のレムはなんの略?英語で答えよ。}
      \item Rapid Eye Movement
\end{enumerate}
\section{つぎの5つの問いに答えよ。提出用書類には番号と答えのみ記載すること。}%8
\begin{enumerate}
      \subsection{特定の事象に時間的に関連して出現する脳の微小電位変化を何と言う?日本語と英語、さらに英略字をすべて答えよ。}
      \item 事象関連電位 (Event-related potential: ERP)
            \subsection{問1の電位変化は非常に小さいため、複数回の試行結果を計算処理して見つける必要がある。そのために用いられる処理方法は何法という?}
      \item 加算平均法
            \subsection{問1の成分のうち、事象の物理特性が引き起こす外因性の電位を何と言う?日本語と英語、さらに英略字をすべて答えよ。}
      \item 誘発電位(evoked potential: EP)
            \subsection{脳波の電位変化は上方向を負として示されることが多い。このとき上むきに現れる波形を何と言う?}
      \item 陰性波
            \subsection{自発的な行為に800ミリ秒程度先立つ、陰性の電位上昇を何と言う?日本語と英語、さらに英略字をすべて答えよ。}
      \item 運動準備電位(Readiness potential: RP)
\end{enumerate}
\section{つぎの4つの問いに答えよ。提出用書類には番号と答えのみ記載すること。}%9
\begin{enumerate}
      \subsection{「ヒトの高度な精神活動のそれぞれの機能系にはある程度の機能局在がある」という考え方を何という?}
      \item 大脳局在論
            \subsection{問1の説を後押しした臨床例をひとつ答えなさい。}
      \item 1848年アメリカの鉄道敷設工事現場で起きた爆発事故で,フィニアス・ゲイジという男性は頭蓋に鉄棒が突き刺さるという事故に遭った.
            奇跡的な回復を見せたものの,「もはや彼はかつての彼ではない」と言われるほど怒りっぽい人格に変わってしまった.
            死後に解剖を行った結果,損傷は前頭葉に限局しており,脳の特定の部位が人格に影響を与えることが示唆された.
            \subsection{ニューロンが活動するとその局所の血流が活動に比例して増加する現象を何という?}
      \item 神経血管結合
            \subsection{強力な均一静磁場の中に高周波の電磁波を照射した直後に一過性に生じる磁化反応の差を、水素原子の挙動に着目して調べる方法を何という?日本語、英語、英略字すべて答えよ。}
      \item 磁気共鳴画像法(Magnetic Resonance Imaging: MRI)
\end{enumerate}
\section{つぎの5つの問いに答えよ。提出用書類には番号と答えのみ記載すること。}%10
\begin{enumerate}
      \subsection{「1つの実験条件を20秒~60秒の時間単位で提示したのち、別の実験条件を同程度の時間提示し、これを繰り返す」といったfMRI研究で用いられる実験デザインを何という?}
      \item ブロックデザイン
            \subsection{問1とともにfMRI研究でよく用いられる、「特定の事象に伴う一過性の信号変化を捉え、その事象に特異的に関連する脳活動を明らかにする」実験デザインを何という?}
      \item 事象関連デザイン
            \subsection{問2の実験デザインは問1の実験デザインに比べて信号検出力が弱い?それとも強い?}
      \item 弱い
            \subsection{脳の体部位局在を明らかにした脳外科医の名前は?}
      \item ペンフィールド
            \subsection{変動磁場により脳内に誘導電流を起こすことで、脳を非侵襲的に刺激する装置を何という?日本語、英語、英略字すべて答えよ。}
      \item 経頭蓋磁気刺激(Transcranial Magnetic Simulation:TMS)
\end{enumerate}
\section{つぎの5つの問いに答えよ。提出用書類には番号と答えのみ記載すること。}%11
\begin{enumerate}
      \subsection{発汗には大きく分けて2種類の機能がある。(1-1)環境温度が高いときに起こる発汗と(1-2)緊張などによって起こる発汗をそれぞれ何という?}
      \item (1-1): 温熱性発汗\\
            (1-2): 精神性発汗
            \subsection{(1-2)の発汗を計測する手法を何という?}
      \item EDA測定
            \subsection{2の手法ではコンダクタンスが用いられることが多い。このとき記録された値の(3-1)持続的な変化と(3-2)一過性の変化をそれぞれ何という?}
      \item (3-1): 水準,レベル\\
            (3-2): 反応,レスポンス
            \subsection{温度調節には(4-1)随意的なものと(4-2)不随意的なものがある。それぞれ何という?}
      \item (4-1): 行動性体温調節\\
            (4-2): 自律性体温調節
            \subsection{2の計測や皮膚温度の計測は自律神経系のうち何の活動を反映している?}
      \item 交感神経活動
\end{enumerate}
\section{つぎの6つの問いに答えよ。提出用書類には番号と答えのみ記載すること。}%12
\begin{enumerate}
      \subsection{心臓全体が規則正しく一体となって収縮するよう調節する心筋線維を何という?}
      \item 刺激伝導系
            \subsection{心臓の活動に起因する電位変化を記録する手法を何という?日本語、英語、英略字すべて答えよ。}
      \item 心電図(Electrocardiogram: ECG)
            \subsection{2の手法で得られる波形のうち、(3-1)最も顕著な波形を何という?また、(3-2)連続するその波形のあいだの時間のことを何という?}
      \item (3-1)QRS波\\
            (3-2)R-R間隔
            \subsection{心臓の拍動間隔にみられるゆらぎを何という?日本語、英語、英略字すべて答えよ。}
      \item 心拍変動(Heart Rate Variability: HRV)
            \subsection{4の解析による時間領域指標のうち、3-2の逆数で表される値を何という?}
      \item 心拍数
            \subsection{4の解析による周波数領域指標では、低周波数成分と高周波数成分がよく算出される。それらのうち副交感神経の活動のみを反映するのはどちら?}
      \item 高周波数成分
\end{enumerate}
% 参考文献
\begin{thebibliography}{99}
      \bibitem{VR}NTTコミュニケーションズ.``VRとは''.\\
      \url{https://www.ntt.com/bizon/glossary/e-v/vr.html}
      \bibitem{AR}NTTコミュニケーションズ.``ARとは''.\\
      \url{https://www.ntt.com/bizon/glossary/e-a/ar.html}
      \bibitem{MR}COCOAR.``ARとは?VR・MRとの違いを分かりやすく解説''.\\
      \url{https://www.coco-ar.jp/media/column/ar-vr-mr/}
      \bibitem{VisionPro}株式会社インプレス.``アップル製品のヘビーユーザーでVRユーザーでもある筆者は「Vision Pro」のココが気になった!''.\\
      \url{https://k-tai.watch.impress.co.jp/docs/review/iphonetips/1506836.html}
\end{thebibliography}
参考文献は2024-7/29に閲覧
\end{document}