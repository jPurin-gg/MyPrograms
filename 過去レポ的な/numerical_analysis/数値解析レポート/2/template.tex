\documentclass[dvipdfmx,titlepage]{jsarticle}
\usepackage[dvipdfmx]{graphicx}
\usepackage{bm}
\usepackage{amsmath}
\usepackage{amssymb}
\usepackage{amsfonts}
\usepackage{comment}
\usepackage{listings}
\usepackage{amsmath}
\usepackage{cases}
\lstset{
    basicstyle={\ttfamily},
    identifierstyle={\small},
    commentstyle={\smallitshape},
    keywordstyle={\small\bfseries},
    ndkeywordstyle={\small},
    stringstyle={\small\ttfamily},
    frame={tb},
    breaklines=true,
    columns=[l]{fullflexible},
    numbers=left,
    xrightmargin=0zw,
    xleftmargin=3zw,
    numberstyle={\scriptsize},
    stepnumber=1,
    numbersep=1zw,
    lineskip=-0.5ex,
    keepspaces=true,
    language=c
}
\renewcommand{\lstlistingname}{リスト}
\makeatletter
\newcommand{\figcaption}[1]{\def\@captype{figure}\caption{#1}}
\newcommand{\tblcaption}[1]{\def\@captype{table}\caption{#1}}
\makeatother
\def\MARU#1{{\rm\ooalign{\hfil\lower.168ex\hbox{#1}\hfil \crcr\mathhexbox20D}}}

\title{\Huge{数値解析レポートNo.2}}
\author{\huge{43番 若月 耕紀}}
\date{}

\begin{document}
\maketitle
\newpage


\section*{\MARU{1}
$
\rm{\bf{A}} = \left(
    \protect\begin{array}{ccc}
      2 & 1 & 1 \\
      1 & -1 & 5 \\
      1 & 2 & -4
    \protect\end{array}
  \right)$ ,
$
\rm{\bf{b}} = \left(
    \protect\begin{array}{c}
      2 \\
      -2 \\
      4
    \protect\end{array}
  \right)$のとき,$\rm{\bf{A}}\rm{\bf{x}}$=$\rm{\bf{b}}$の解$\rm{\bf{x}}$を求めよ.
}

授業課題で作成したLU分解のプログラムを使用すると,LU分解の結果は以下のようになる.
\begin{equation}
	L = \left(
    \protect\begin{array}{ccc}
      1 & 0 & 0 \\
      0.5 & 1 & 0 \\
      0.5 & -1 & 1
    \protect\end{array}
  \right),
	U = \left(
    \protect\begin{array}{ccc}
      2 & 1 & 1 \\
      0 & -1.5 & 4.5 \\
      0 & 0 & 0
    \protect\end{array}
  \right) \nonumber
\end{equation}

この時,U行列のランクが2となるので,解である$x$を求める際に0割りをすることになり,プログラムが強制終了する.対策方法としては,0割りが出現しないような行列を入力すれば良い.

また解$x$は行基本変形を用いて以下のように求めることができる.

\begin{equation*}
  \left(
    \protect\begin{array}{ccc|c}
      2 & 1 & 1 & 2 \\
      1 & -1 & 5 & -2 \\
      1 & 2 & -4 & 4
    \protect\end{array}
  \right)\rightarrow
  \left(
    \protect\begin{array}{ccc|c}
      1 & 0 & 2 & 0 \\
      0 & 1 & -3 & 2 \\
      0 & 0 & 0 & 0
    \protect\end{array}
  \right),
  x= \left(
    \protect\begin{array}{ccc|c}
      -2t \\
		3t+2 \\
		t
    \protect\end{array}
  \right)
 \left(
    tは任意定数
  \right)
\end{equation*}

今回は不定解となるため,$t$を任意定数として計算している.

\section*{\MARU{2}
$
\rm{\bf{C}} = \left(
    \protect\begin{array}{ccc}
      2 & -1 & 5 \\
      -4 & 2 & 1 \\
      8 & 2 & -1
    \protect\end{array}
  \right)$の逆行列を求めよ.
}

CがLU分解できるとき,$C^{-1}=U^{-1}L^{-1}P$と表すことができる.よって,$L$と$U$の逆行列を求めれば,$C$の逆行列を求めることができる.

\subsection*{$U$の逆行列}
$U$は上三角行列であるので,$U^{-1}$も上三角行列である.これらを以下のように定める.

\begin{equation*}
	U = \left(
	\begin{array}{ccc}
		a_{11} & a_{12} & a_{13} \\
		0 & a_{22} & a_{23} \\
		0 & 0 & a_{33}
	\end{array}
	\right), \
	U^{-1} = \left(
	\begin{array}{ccc}
		a_{11}^{\prime} & a_{12}^{\prime} & a_{13}^{\prime} \\
		0 & a_{22}^{\prime} & a_{23}^{\prime} \\
		0 & 0 & a_{33}^{\prime}
	\end{array}
	\right)
\end{equation*}

$UU^{-1}=E$であるため,

\begin{equation*}
	UU^{-1} = \left(
	\begin{array}{ccc}
		a_{11}a_{11}^{\prime} & a_{11}a_{12}^{\prime} + a_{12}a_{22}^{\prime} & a_{11}a_{13}^{\prime} + a_{12}a_{23}^{\prime} + a_{13}a_{33}^{\prime} \\
		0 & a_{22}a_{22}^{\prime} & a_{22}a_{23}^{\prime} + a_{23}a_{33}^{\prime} \\
		0 & 0 & a_{33}a_{33}^{\prime}
	\end{array}
	\right) = \left(
	\begin{array}{ccc}
		1 & 0 & 0 \\
		0 & 1 & 0 \\
		0 & 0 & 1
	\end{array}
	\right)
\end{equation*}

となる.これを解くと,

\begin{equation*}
	U^{-1} = \left(
	\begin{array}{ccc}
		a_{11}^{\prime} & -a_{12}a_{11}^{\prime}a_{22}^{\prime} & -a_{12}a_{11}^{\prime}a_{23}^{\prime} - a_{13}a_{11}^{\prime}a_{33}^{\prime} \\
		0 & a_{22}^{\prime} & -a_{23}a_{22}^{\prime}a_{33}^{\prime} \\
		0 & 0 & a_{33}^{\prime}
	\end{array}
	\right)
\end{equation*}

となる.この時,
	$\displaystyle a_{11}^{\prime} = \frac{1}{a_{11}},
	a_{22}^{\prime} = \frac{1}{a_{22}},
	a_{33}^{\prime} = \frac{1}{a_{33}}$である.

\subsection*{$L$の逆行列}
$U$と同様に以下のように定める.

\begin{equation*}
	L = \left(
	\begin{array}{ccc}
		b_{11} & 0 & 0 \\
		b_{21} & b_{22} & 0 \\
		b_{31} & b_{32} & b_{33}
	\end{array}
	\right), \
	L^{-1} = \left(
	\begin{array}{ccc}
		b_{11}^{\prime} & 0 & 0 \\
		b_{21}^{\prime} & b_{22}^{\prime} & 0 \\
		b_{31}^{\prime} & b_{32}^{\prime} & b_{33}^{\prime}
	\end{array}
	\right)
\end{equation*}

$LL^{-1}=E$であるので,
                
\begin{equation*}
	LL^{-1} = \left(
	\begin{array}{ccc}
		b_{11}b_{11}^{\prime} & 0 & 0 \\
		b_{21}b_{11}^{\prime} + b_{22}b_{21}^{\prime} & b_{22}b_{22}^{\prime} & 0 \\
		b_{31}b_{11}^{\prime} + b_{32}b_{21}^{\prime} + b_{33}b_{31}^{\prime} & b_{32}b_{22}^{\prime} + b_{33}b_{32}^{\prime} & a_{33}a_{33}^{\prime}
	\end{array}
	\right) = \left(
	\begin{array}{ccc}
		1 & 0 & 0 \\
		0 & 1 & 0 \\
		0 & 0 & 1
	\end{array}
	\right)
\end{equation*}

となる.これを解くと,

\begin{equation*}
	L^{-1} = \left(
	\begin{array}{ccc}
		b_{11}^{\prime} & 0 & 0 \\
		-b_{21}b_{11}^{\prime}b_{22}^{\prime} & b_{22}^{\prime} & 0 \\
		-b_{31}b_{11}^{\prime}b_{33}^{\prime} - b_{32}b_{21}^{\prime}b_{33}^{\prime} & -b_{32}b_{22}^{\prime}b_{33}^{\prime} & b_{33}^{\prime}
	\end{array}
	\right)
\end{equation*}

となる.この時,$\displaystyle
	b_{11}^{\prime} = \frac{1}{b_{11}},
	b_{22}^{\prime} = \frac{1}{b_{22}},
	b_{33}^{\prime} = \frac{1}{b_{33}}$である.

\subsection*{$C$の逆行列}
課題5で作成したプログラムでCをLU分解すると以下のようになる.

\begin{equation*}
	L = \left(
	\begin{array}{ccc}
		1 & 0 & 0 \\
		-0.5 & 1 & 0 \\
		0.25 & -0.5 & 1
	\end{array}
	\right), \
	U = \left(
	\begin{array}{ccc}
		8 & 2 & -1 \\
		0 & 3 & 0.5 \\
		0 & 0 & 5.5
	\end{array}
	\right)
\end{equation*}

上記を利用して,$L$と$U$の逆行列を求めると以下のようになる.

\begin{equation*}
	L^{-1} = \left(
	\begin{array}{ccc}
		1 & 0 & 0 \\
		0.5 & 1 & 0 \\
		0 & 0.5 & 1
	\end{array}
	\right), \
	U^{-1} = \left(
	\begin{array}{ccc}
		0.125 & -0.08333 & 0.3030 \\
		0 & 0.3333 & -0.03030 \\
		0 & 0 & 0.1818
	\end{array}
	\right)
\end{equation*}

これらを利用して$C^{-1}=U^{-1}L^{-1}P$を計算すると以下のようになる.

\begin{equation*}
	C^{-1}=U^{-1}L^{-1}P = \left(
	\begin{array}{ccc}
		0.3030 & -0.06818 & 0.08333 \\
		-0.03030 & 0.3182 & 0.1667 \\
		0.1818 & 0.09090 & 0
	\end{array}
	\right)
\end{equation*}

\setcounter{section}{1}
\section{2分法とよく似た方法にはさみうち法がある.はさみうち法について調査し,2分法との違いをまとめよ.また,同じ条件で10回繰り返した時の値について比較せよ.}

はさみうち法とは,2分法を少し改良したものである.2分法では区間の中間を計算していたが,はさみうち法では区間の端点を結ぶ直線とx軸との交点の座標を使用する.はさみうち法は,初期値が悪いと2分法よりも収束が遅くなるが,初期値が解に近ければ収束は早くなる.

はさみうちについて,作成したプログラムをリスト\ref{src:2}に示す.

\begin{lstlisting}[caption=False\_Position, label=src:2]
void False_Position(double a1, double a2, double (*f)(double x)){
	int i;
	double nex;

	for(i = 1; 10 >= i; ++i){
		nex = (a1 * f(a2) - a2 * f(a1)) / (f(a2) - f(a1));

		printf("%2d: x=%.9f\n", i, nex);

		if(0 > f(a1) * f(nex)){
			a2 = nex;
		}else{
			a1 = nex;
		}
	}
}
\end{lstlisting}

基本的なプログラムの構成は課題3と同じであるため,説明を省略する.変更点であるnexについては,x軸を$0\cdot x + 1 \cdot y + 0 = 0$として交点の座標を求めた.

このプログラムを用いて$f(x)=x^4-3x^2+5x-9$,区間[1,3]の解を求める.実行した結果を以下に示す.

\begin{verbatim}
はさみうち法          2分法
1:  x = 1.181818182     1:  x = 2.000000000
2:  x = 1.330162049     2:  x = 1.500000000
3:  x = 1.447303924     3:  x = 1.750000000
4:  x = 1.536568968     4:  x = 1.875000000
5:  x = 1.602390729     5:  x = 1.812500000
6:  x = 1.649612800     6:  x = 1.781250000
7:  x = 1.682775076     7:  x = 1.765625000
8:  x = 1.705697489     8:  x = 1.757812500
9:  x = 1.721362780     9:  x = 1.753906250
10: x = 1.731983564     10: x = 1.751953125
\end{verbatim}

区間[1,3]での解は$x\approx 1.753663501$であるため,今回の場合では,はさみうち法の方が収束が遅かった.

\section{ヤコビ法やガウス・ザイデル法が収束しない条件を,
例をあげながら解説せよ.}

ヤコビ法,ガウス・ザイデル法とは,n元連立一次方程式を反復法で解く手法の1つである.これらが真の解に収束する条件は以下のようになる.
\begin{itemize}
 \item 第$i$行の対角項以外の成分の絶対値の和よりも,対角項の絶対値が大きい
\end{itemize}
上記の条件を満たさなかった場合,ヤコビ法とガウス・ザイデル法が収束しない.よって収束しない条件は以下のようになる.
\begin{itemize}
 \item 第$i$行の対角項以外の成分の絶対値の和よりも,対角項の絶対値が小さい
\end{itemize}

例として,以下のように定めた場合について各方法の解の遷移を示す.

\begin{equation*}
	\rm{\bf{A}} = \left(
	\begin{array}{ccc}
		1 & 2 & 3 \\
		3 & 1 & 2 \\
		2 & 3 & 1
	\end{array}
	\right), \
	\rm{\bf{b}} = \left(
	\begin{array}{c}
		14 \\
		11 \\
		11
	\end{array}
	\right), \
	\rm{\bf{初期値}} = \left(
	\begin{array}{c}
		0 \\
		0 \\
		0
	\end{array}
	\right), \
	\rm{\bf{正しい解}} = \left(
	\begin{array}{c}
		1 \\
		2 \\
		3
	\end{array}
	\right)
\end{equation*}
まず,ヤコビ法での解の遷移を以下に示す.
\begin{equation*}
            \left(
                \begin{array}{c}
                    0 \\
                    0 \\
                    0
                \end{array}
            \right) \rightarrow \left(
                \begin{array}{c}
                    14 \\
                    11 \\
                    11
                \end{array}
            \right) \rightarrow \left(
                \begin{array}{c}
                    -41 \\
                    -53 \\
                    -50
                \end{array}
            \right) \rightarrow \left(
                \begin{array}{c}
                    270 \\
                    234 \\
                    252
                \end{array}
            \right) \rightarrow \left(
                \begin{array}{c}
                    -1210 \\
                    -1303 \\
                    -1231
                \end{array}
            \right)\rightarrow・・・
        \end{equation*}
次に,ガウス・ザイデル法での解の遷移を以下に示す.
\begin{equation*}
            \left(
                \begin{array}{c}
                    0 \\
                    0 \\
                    0
                \end{array}
            \right) \rightarrow \left(
                \begin{array}{c}
                    14 \\
                    -31 \\
                    76
                \end{array}
            \right) \rightarrow \left(
                \begin{array}{c}
                    -152 \\
                    315 \\
                    -630
                \end{array}
            \right) \rightarrow \left(
                \begin{array}{c}
                    1274 \\
                    -2551 \\
                    5116
                \end{array}
            \right) \rightarrow \left(
                \begin{array}{c}
                    -10232 \\
                    -20475 \\
                    -40950
                \end{array}
            \right)\rightarrow・・・
        \end{equation*}
上記の結果より,収束条件を満たさない場合,どちらの方法でも,解が収束せずに発散していくことがわかる.

\section{感想}
今回のレポートでは,今まで習ったことを改めて学習することができた.授業課題として実装していないものもあったため,少し忘れている部分があり,苦労した.これから編入試験もすることになるため,課題として出ていなくても積極的に学習をし,理解を深めていきたいと思う.

\end{document}