\documentclass[titlepage]{jarticle}
\begin{document}
\section*{五年選択科目のすゝめ}
楽単度、推奨度、成長性の観点から5段階で評価をする。あくまでも独断と偏見によっているので参考程度に押えて欲しい。
それぞれこんな印象で設定してある。

楽単度5→余程のことがなければ単位はある

推奨度5→日程がクソでも取るべき

成長性5→凄く力が付く



楽単度1→かなり頑張らないと単位はない

推奨度1→よほど日程が良くても取りたくはない

成長性1→自分で勉強したほうがマシ

\section*{一般科目1群}
すべて一単位で前期のみのもの.
ここだけとっても単位のくっつきが悪いので単位目的で取るのはおすすめしない.
おすすめの選び方は興味があることを突き詰めて,受験勉強の息抜き程度に考えるべき.
ちなみに筆者は取っていない.

\subsection*{経済学}
楽単度?
推奨度2
成長性?

正直取ってない上に情報も集めなかったのであまりわからない.
統計云々の話をするみたいな話をしてた気がする.
\subsection*{哲学}
楽単度?
推奨度2
成長性?

倫理の進化版みたいな感じらしい.
倫理大好きっ子だった人は受けてもいいかもしれない.

\subsection*{歴史学}
楽単度?
推奨度3
成長性?

歴史の教科書を見比べるみたいなやつらしい.
話を聞いたときに唯一取らなかったのを後悔した.
右寄りの教科書と左寄りの教科書を見比べてこんな違いがあるんだ~.ふ~んってなるやつ.
筆者は面白そうだと思った.

\section*{一般科目2群}
決めるときに配られた紙を見るのが一番わかり易いのでそっちを参考にしてほしい.
すべてTOEIC470点で単位認定されるのでTOEIC頑張れ.
英語VDの技科大英語の対策のやつはまじで長岡技科大受ける人しか意味ないので,これ以外の人は譲ってあげてほしい.

\section*{一般科目3群}
第二外国語を受けるやつ.通年.一応特に言うことはないけど,ドイツ語受けてる人はそのままドイツ語でいい.
あと,技科大に行ったとき,中国語受けてた人は中国語取れず,韓国語受けてた人は韓国語受けれないらしい.
一応留意しておくといい.

\section*{線形制御}
楽単度4
推奨度5
成長性4

外山先生の授業.行列関係の総復習をする感じ.受験する人は黙って受けとけ.
過去問より授業毎の小テストを使って勉強すべし.
\section*{制御工学2}
楽単度4
推奨度4
成長性3

たくしの授業.基本テキストが配られて黙々と作業をする科目.単位の付け方は課題レポート出して終わり.
課題レポートをすべて埋めて出せば基本的に単位が生える.
6,7割の問題だけで良いよ~とか言ってくるが,この甘い誘惑に負けた人は基本的に単位を落としている.
\section*{センサー工学}
楽単度5
推奨度4
成長性3

梅田の授業.一人一つセンサについて調べて発表をする科目.
レジュメ,発表スライド,発表とおそらく擬似的な卒研発表の前段階として設けられている.
しかし,受験のタイミングとレジュメ提出などが重なったりして大変そうだった.
また,評価の決め方的に,発表させたにも関わらずテストがあってカス.
基本みんなが取る科目らしいが,この理由から筆者は取っていない.
\section*{材料力学2}
楽単度3
推奨度2
成長性3

永井の授業.弾性体についての授業.物体がたわんだときにどんな力がかかるかみたいなやつ.
内容自体は物理っぽいので他の授業があったら受けてもいいかもなやつ.
ただ,基本1,2限にしかなさそうなので多分誰も取らなさそう.
ちなみに筆者は取っていない.

\section*{データ通信工学}
楽単度4
推奨度5
成長性4

情報系の道に進むなら絶対に取るべき.sky先生の授業.
多分これ取らないと編入先でついていけないってくらい基本的なことをする.
筆者の代のテストでは某車修理会社をいじる問題が出ていた.
5年生相手だからかsky先生もはっちゃけ始めるので面白い.黙ってこれを受講しろ.

\section*{コンピュータネットワーク}
楽単度5
推奨度5
成長性3

情報系の道n(以下略).竹部の授業.普通受けても損はない.テストは持ち込み可なので単位としても楽.
ただ,専門科目にしては珍しく1単位なので注意が必要.
\section*{ロボット工学}
楽単度4
推奨度4
成長性3

たくしのことが好きかどうかで決めてもいいかも.
単位が足りていれば取る必要はなくなったとか言ってみんないなくなる.ちょっと可愛そう.
持ち込みありのテストだけど授業を聞いていないとついていけない.
あと,一回の授業が5,6,7と三時間使って1月以降授業がなくなる.研究に集中できるねとのことらしい.
\section*{ネットワークプログラミング}
楽単度4
推奨度5
成長性3

竹部の授業.CとJAVAでネットに接続し始める.最終的にはbotをつくって対戦させる.テストはある.
\section*{物理学2B}
楽単度4
推奨度3
成長性4

(くっそ優しい)シュウ酸の授業.点数の決め方は波動に関する小テスト3つと課題が少々だけだが,物理学全般の拾いきれなかったとことか言っていろんなことをする.
やや圧はあるが,ちゃんと対話が可能.だいたいのことなら許してくれる.
ラーメン食いに行ってて出れません(意訳)みたいな理由の欠席も笑って許してくれた.(なんで)

\section*{プログラミング演習2}
楽単度5
推奨度5
成長性5

高橋研の先生による授業.夏休み中に行う.OpenGLが使えるようになる.一応カスみたいなレポートと課題でも出せば単位が出る(先生談).
対話は可能で,演劇などに理解があるので期限がヤバかったら早めに相談に行こう.
\end{document}