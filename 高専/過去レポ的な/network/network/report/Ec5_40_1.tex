\documentclass{jsarticle}
\usepackage[dvipdfmx]{graphicx}
\usepackage{bm}
\usepackage{amsmath}
\usepackage{amssymb}
\usepackage{amsfonts}
\usepackage{comment}
\usepackage{listings}
\usepackage{cases}
\usepackage{siunitx}
\usepackage[hyphens]{url}
\usepackage{listings}
\usepackage{jlisting}
\lstset{
    basicstyle={\ttfamily},
    identifierstyle={\small},
    commentstyle={\smallitshape},
    keywordstyle={\small\bfseries},
    ndkeywordstyle={\small},
    stringstyle={\small\ttfamily},
    frame={tb},
    breaklines=true,
    columns=[l]{fullflexible},
    numbers=left,
    xrightmargin=0zw,
    xleftmargin=3zw,
    numberstyle={\scriptsize},
    stepnumber=1,
    numbersep=1zw,
    lineskip=-0.5ex,
    keepspaces=true,
    language=c
}
\renewcommand{\lstlistingname}{リスト}
\makeatletter
\newcommand{\figcaption}[1]{\def\@captype{figure}\caption{#1}}
\newcommand{\tblcaption}[1]{\def\@captype{table}\caption{#1}}
\makeatother

\title{\vspace{-3cm}ネットワークプログラミング レポート}
\author{Ec5 40番 若月 耕紀}
\date{}

\begin{document}
\maketitle

\section{はじめに}
本レポートでは,ネットワークプログラミングのロボットプログラムについての動作を記す.

\section{プログラムの動作}
作成したロボットプログラムの動作について以下に示す.

\subsection{自身やエネルギータンクの情報取得}

目標とするエネルギータンクを決めるために,自身の位置情報と,フィールドのエネルギータンクの情報を取得する.取得する情報は,自身の座標,各エネルギータンクの座標と得点とする.

\subsection{目標の決定}

今回作成したプログラムでは,基本的には自身に最も近いエネルギータンクを目標とする.ただし,2点間の距離を算出する際に,エネルギータンクのポイントに応じて距離を短くするように減算を行う.また,境界を越えると反対側に出ることを考慮して計算する.

\subsection{移動について}

自身の座標とエネルギータンクの座標を長方形(正方形)の対角の頂点として考える.この時,長方形に近い形であれば,上下または左右に大きく移動させ,正方形に近い形であれば,斜めに移動させる.

移動する方向を決める際には,目標を決定する際と同様に,境界を越えたら反対側に出ることを考慮する.

\end{document}