\documentclass[dvipdfmx]{jsarticle}
\usepackage{amsmath}
\usepackage{mathtools}
\usepackage[dvipdfmx]{graphicx}
\usepackage{float}
\usepackage{here}

\begin{document}
\title{R5電制祭 企画書}
\author{EC5 35番 本間 三暉}
\maketitle

\section{目的}
\begin{itemize}
  \item 交千祭において5位だった結果を振り返り,11月にある学園祭演劇ではより良い結果を残せるように電子制御工学科の学生間の交流を深める.
  \item 上学年の学生が下学年の学生と進んでコミュニケーションを取り,進学・就職・研究室に対する下学年の関心や意欲を高める.
\end{itemize}

\section{開催日時・場所}
\begin{description}
  \item[日時:] (要相談)夏休み明けの特別授業日が好ましい
  \item[場所:] (要相談)第1体育館・第2体育館
\end{description}
\section{参加者}
電子制御工学科1~3,5年 計160人 (4年生は研修旅行のため不参加)

電子制御工学科教員 数名


参加者を\ref{team}節の通りに8チームに分け,各学年の人数は均等にする.
\section{プログラム}
電制祭の大まかなタイムスケジュールを表\ref{time}に示す.
\begin{table}[H]

  \caption{タイムテーブル}
  \label{time}
  \centering
  \begin{tabular}{l||l|l}
          & 第一体育館        & 第二体育館       \\\hline\hline
    12:40 & 集合・整列        &             \\
    12:50 & 開会式          &             \\
    13:10 & アイスブレイク      &             \\
    13:40 & ドッジボール αブロック & ドッジボールβブロック \\
    14:35 & 休憩           &             \\
    14:50 & 準決勝1         &             \\
    15:00 & 準決勝2         &             \\
    15:10 & 決勝           &             \\
    15:20 & 閉会式          &
  \end{tabular}
\end{table}

% スポーツ種目はドッジボールを行う.チーム数は8チームとし,各学年の人数は均等にな
% るようにする.予選リーグと決勝トーナメントを行い,予選リーグは2ブロックに分けて行
% い,各ブロックの勝利数上位2チームが決勝トーナメントに進む.決勝トーナメントはく
% じで試合相手を決め,トーナメントを行う.決勝試合を行う前に予選敗退となった4チー
% ムで交流戦を行う.各チーム最低でも 4 試合行えるようにする.

\section{アイスブレイク}
開会式後に各チームごとに分かれアイスブレイクを行う.
内容はまだ決まってないです.追々考えます.


\section{ドッジボール}
\subsection{概要}
\begin{itemize}
  \item チーム数は8チームとする.
  \item 各学年の人数を均等になるようにする.
  \item 予選リーグと決勝トーナメントを行い,予選リーグは2つのブロックに分けて行う.
  \item 各ブロックの勝利数上位2チームが決勝トーナメントに進む.
  \item 決勝トーナメントはくじで試合相手を決める.
\end{itemize}

\subsection{ルール}
\begin{itemize}
  \item ボールはソフトバレーボールを1つ使用する.
  \item 試合開始時に外野に3名までおいてよい.(おかなくても良いが,外野に誰もいない場合ボールに近い側のチームに所有権があるとする)
  \item 外野の選手が相手チームの内野の選手をアウトにした場合は{\bf 必ず内野に復帰する}.
  \item 試合開始前,内野にいるときにボールに被弾する,外野にいるときに相手チームの内野の選手をアウトにする以外のタイミングでの内野と外野に行き来は認めない.
  \item 5年生はみんなにボールが回るように意識する.
  \item {\bf 試合後,内野にいる女子1人につき1.5人と数える}.(今年は5年生女子の人数が多いため例年より女子の重みを減らす)
  \item 引き分けの場合は各チーム代表者5名でじゃんけんを行い,先に3回勝った方の勝利とする.
  \item チームの人数に差があった場合,人数が多い方に揃える.足りない分は学年の人数に差がでないことを心がけ,
        周りにいる体力が有り余っている学生を参加させる.その学生はチーム内で話し合って決める.
  \item 試合中のメンバーの増減は認めない.試合の参加者は試合開始時に整列したメンバーで確定する.
  \item その他,本番中に不都合が起きた場合や下級生の楽しみが著しく損なわれる場合は審判の判断によってルールの追加や改変を認める.
\end{itemize}

\subsection{チーム分け}\label{team}
\begin{itemize}
  \item 1~3年生は各クラスごとにくじ引きでチームを決定する.
  \item 5年生は様々なバランスを考慮しチームを決定する.
  \item 昨年と同様,夏休みに入る前の特別活動と実験の時間を焼く10分お借りして行う.
  \item 教職員は固定したチームに所属せず,ご都合に合わせて随時試合が始まるチームに加わっていただく.
\end{itemize}

\section{物品借用}
電制祭で使用する物品を表\ref{buppin}に示す.
\begin{table}[H]
  \caption{使用物品一覧}
  \label{buppin}
  \centering
  \begin{tabular}{l|lcl}
    物品名       & 数量     & 管理者  & 備考                   \\\hline\hline
    ソフトドッジボール & 3個     & 酒井先生 &                      \\
    カラーテープ    & 160本作成 & 制御演劇 & 学年が一目でわかるように腕に巻いてもらう \\
    電子ホイッスル   & 2個     & 学生課  &                      \\
    マイク・スピーカー & 1セット   & 学生課  &                      \\
    大型タイマー    & 2台     & 体育科  & 第一体育館,第二体育館にそれぞれ一つずつ \\
    三角コーン     & 12~15個 & 体育科  &
  \end{tabular}
\end{table}


\end{document}