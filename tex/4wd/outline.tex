% \documentclass[twoside,twocolumn]{ujarticle}
\documentclass[titlepage]{jarticle}
%\usepackage{type1cm}
\usepackage{outline-ec}
\usepackage{amsmath,amssymb,verbatim,ascmac,multicol}
\usepackage{tabularx}
% dvioutで確認する場合は以下を有効にする
%\usepackage[dviout]{graphicx,color}
% pdf化する場合は以下を有効にする
\usepackage[dvipdfmx]{graphicx,color}

%
% --------------------------------------------------------------------------
% 図表番号の後の:を削除
%
\makeatletter
\long\def\@makecaption#1#2{% #1=図表番号、#2=キャプション本文
\sbox\@tempboxa{#1 \hskip0.5zw #2}% 図表番号とキャプションの間のスペース 0.5zw
\ifdim \wd\@tempboxa >\hsize
#1 #2\par 
\else
\hb@xt@\hsize{\hfil\box\@tempboxa\hfil}
\fi}
\makeatother
% --------------------------------------------------------------------------

%
% --------------------------------------------------------------------------
% 下の該当する部分を書き換える
%
\氏名{本間 三暉}			%% 自分の氏名
\出席番号{35}					%% 出席番号
\研究室名{視覚情報処理研究室}			%% 研究室名
\指導教官{高橋 章}			%% 指導教員名
%
% --------------------------------------------------------------------------
% 研究題目等
%
% 研究題目が2行になるときは \\ で行送りできる
%
\発表番号{B\;--\;}
\研究題目{モーションキャプチャデバイスと画像処理を利用したバランス運動の解析}

% --------------------------------------------------------------------------
% アブストラクト
%
\アブストラクト{
モーションキャプチャデバイスmocopiを,運動解析に活用することを検討する.
そのため,竹馬や一輪車などの精密な重心移動や姿勢制御が必要な運動に関して測定を行う.
しかし,モーションキャプチャデバイスだけでは肘や膝などの関節の屈曲を正確に計測することができないため,
画像処理により骨格推定を組み合わせ,バランス運動時の重心や姿勢を解析する.
}
%
% --------------------------------------------------------------------------
% 本文開始
%
\begin{document}
\maketitle

%
% --------------------------------------------------------------------------
% 第1節
%
\section{研究背景・目的}
%
情報通信技術の急速な進歩により人工現実感,拡張現実感,複合現実感などの応用が広がっている.
感染症対策を契機にオンラインコミュニケーションも増加し,インターネット上の仮想共有空間であるメタバースが注目されている.
三次元の仮想空間で自分の分身となるアバターを自由に操作するには,
体の動きを計測する必要があり,画像処理による方法や専用デバイスを装着する方法などが試みられている. 

本研究では市販のモーションキャプチャデバイスmocopiを,運動解析に活用することを検討する.
このデバイスは両手,両足,頭,腰の計6か所に小型センサを装着してリアルタイムに三次元計測を行うことができるが,肘や膝などの関節の屈曲を正確に計測することができない.
そこで画像処理による骨格推定を組み合わせ,一輪車や竹馬のような器具を使うバランス運動の動作解析を実現する.
これにより,身体が身体自信や使用器具などに隠れてしまう場合の骨格推定の精度低下の解消,三次元計測の精度の向上,計測速度のさらなる改善などの問題を解決することが出来る.
そして、スポーツや映像作品などの様々な分野で,使い方が限定されていたモーションキャプチャの応用範囲が広がることが期待できる.
%
% --------------------------------------------------------------------------
% 第2節
%
\section{研究内容・目的}
%
モーションキャプチャデバイスmocopiと画像処理による骨格推定を組み合わせ,
それぞれの測定方法の欠点となりうる部分を補いつつ,バランス運動を解析する.
%
% --------------------------------------------------------------------------
% 第2節 第1小節
%
\subsection{開発環境}
%
本研究では,開発環境としてmocopiとOpenPoseを使用する.
%
% --------------------------------------------------------------------------
% 第2節 第2小節
%
\subsection{発表者氏名および所属研究室名等}
%
発表者氏名および所属研究室名等は明朝体,10.5 pt で記載してください.
研究室名と指導教員名はカッコ内に記載してください.
%
% --------------------------------------------------------------------------
% 第2節 第3小節
%
\subsection{Abstract}
%

%
% --------------------------------------------------------------------------
% 第2節 第4小節
%
\subsection{本文}
%
本文は,日本語は明朝体,10.5 ptで記載してください.
英語や記号など半角文字を使用する場合は,Serif体, 10.5 ptで記載してください.
%
% --------------------------------------------------------------------------
% 第2節 第5小節
%
\subsection{参考文献について}
%
%本文中の参考文献は,引用元を示す場合には上付きで明記(○○\cite{okumura})し,
%本文中で用いる場合には,本文のフォントサイズと同じサイズ(文献\Cite{okumura}より)で記載してください.

参考文献リストでは,基本的に本文より若干小さいフォントで記載してください.
日本語は明朝体, 10 pt,英語は Serif体,9〜10 pt としてください.
リスト番号は下記のように記載してください.
%
\renewcommand{\labelenumi}{[\arabic{enumi}]}
\begin{enumerate}
  \small{
  \item 奥村晴彦, ``改訂第 3 版 \LaTeX2e 美文書作成入門'', 技術評論社, 2004
  \item 藤田眞作,``\LaTeXe 階梯 第 2 版'',ピアゾン,2000
        }
\end{enumerate}
\renewcommand{\labelenumi}{\arabic{enumi}}
%
%
% --------------------------------------------------------------------------
% 第2節 第6小節
%
\subsection{図表}
%
図表のキャプションは,図○,表○としてください.
図表を挿入したら必ず本文中で「○○を 図 5 に示す」というような説明文を加えてください.
%
\begin{table*}[t!]
  \centering
  \caption{○○}
  \framebox(450,100){}
  \label{tab:ex_tab1}
\end{table*}
%
%
% --------------------------------------------------------------------------
% 第3節
%
\section{予稿の書き方}
%
予稿は,研究内容がわからない人が読むということを意識して書いてください.
ただし,スペースの関係もあるので,省略するところは省略し,
研究の概要がわかるようにまとめてください.
%
% --------------------------------------------------------------------------
% 第3節 第1小節
%
\subsection{予稿に記載する事項}
%
予稿は以下の項目を参考に構成してください.
%
\begin{enumerate}
  \item \textbf{研究背景・目的} \\
        自分が進めようとする研究に関して,現在または過去の事実を述べ,
        ニーズや問題点を挙げるなどして,研究目的を述べてください.
        場合によっては長期的な目標(最終目標)があり,
        そのためのアプローチとしての短期目標(本年度の目的)があるかもしれませんので,
        その点も明記してください.
  \item \textbf{研究(または実験)内容・方法} \\
        自分の研究の核となる理論,方法,装置などの説明を述べてください.
        場合によっては節,小節を設けて述べていく必要もあろうかと思います.
  \item \textbf{研究(または実験)結果} \\
        自分の研究で得られた結果を記載してください.
        データなどを測定している場合には,その測定方法などを説明し,
        結果やその考察などを載せてください.
        考察では,その結果からどんなことが言えるのか,
        その結果は何かに裏付けされているのかなどを示してください.
  \item \textbf{まとめ・今後の課題} \\
        自分の研究のまとめと今後の課題を簡素に記載してください.
        特に,研究結果では,何が得られたのか,どういうことが分かったのかを記載してください.
  \item \textbf{参考文献} \\
        研究を進めていくうえで参考にした参考文献を挙げてください.

        参考文献のフォーマットとしては,
        %
        \renewcommand{\labelenumii}{[\arabic{enumii}]}
        \begin{enumerate}
          \small{
          \item 本の著者,``本の題名'',出版社,発行年
          \item 論文の著者 1,論文の著者 2,``論文の題目'',論文誌名,巻,号,掲載ページ,発行年
                }
        \end{enumerate}
        \renewcommand{\labelenumii}{\arabic{enumii}}
        %
        としてください.
\end{enumerate}
%
% --------------------------------------------------------------------------
% 第3節 第2小節
%
\subsection{\TeX 版での設定項目}
%
原稿執筆に際して,サンプルファイル(\texttt{outline.tex})で,
はじめに設定する項目について説明します.

以下の項目を必ず設定してください.
研究室記号は変更になる可能性があるので,よく確認して間違いのないようにしてください.
特に,発表番号は研究室内での発表順になるので,指導教員とすり合わせを行っておいてください.
%
\begin{quote}
  \verb|\氏名{著者名}| \\
  \verb|\出席番号{出席番号}| \\
  \verb|\発表番号{研究室記号\;--\;発表番号}| \\
  \verb|\研究題目{研究題目}|
\end{quote}
%
卒研発表会の予稿では,アブストラクトが必須になりますので,以下の項目を設定してください.
%
\begin{quote}
  \verb|\アブストラクト{アブストラクト}|
\end{quote}
%
%
% --------------------------------------------------------------------------
% 第3節 第3小節
%
\subsection{本文中の文献参照について}
%
参考文献を引用するには,\verb|\cite{文献ラベル}|とすることで,
本文中に文献リスト番号を出力してくれます.

本文として文献番号を用いたい場合には,\verb|\Cite{文献ラベル}|のように,
通常の文献参照コマンドの頭文字を大文字に変えたコマンドを利用することで
本文と同じサイズで表示されます.
%
% --------------------------------------------------------------------------
% 第3節 第4小節
%
\subsection{図表について}
%
予稿は 2 段組で構成されているので,図表の作成には注意してください.
図表の横幅は,基本的に本文の横幅よりも少し狭いように作成し,
図表内で使用する文字サイズはできるだけ本文の文字サイズよりも一回り小さなサイズ設定としてください.
また,小さすぎる文字サイズは避けるようにしてください.
%
\begin{figure}[h!]
  \centering
  \framebox(200,80){}
  \caption{○○}
  \label{fig:ex_fig1}
\end{figure}
%
%
\begin{figure}[h!]
  \centering
  \framebox(200,90){}
  \caption{○○}
  \label{fig:ex_fig2}
\end{figure}
%
%
% --------------------------------------------------------------------------
% 第3節
%
\section{おわりに}
本稿では \TeX を用いた卒業研究発表会用の予稿原稿の書き方について説明しました.
\TeX 版では,初期設定項目や特殊な使い方がありますので,本稿を熟読の上,原稿を執筆してください.
%
% --------------------------------------------------------------------------
% 参考文献
%

\begin{thebibliography}{99}
  \small{
    \bibitem{okumura}{
      奥村晴彦, ``改訂第 3 版 \LaTeX2e 美文書作成入門'', 技術評論社, 2004
    }
    \bibitem{fujita}{
      藤田眞作,``\LaTeXe 階梯 第 2 版'',ピアゾン,2000
    }
    \bibitem{takahashi}{
      高橋章, ``\TeX によるレポート作成'',
      電子制御工学科 第 3 学年前期学生実験テキスト, 2003
    }
  }
\end{thebibliography}


\end{document}
