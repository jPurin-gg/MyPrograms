% \documentclass[twoside,twocolumn]{ujarticle}
\documentclass[titlepage]{jarticle}
%\usepackage{type1cm}
\usepackage{outline-ec}
\usepackage{amsmath,amssymb,verbatim,ascmac,multicol}
\usepackage{tabularx}
% dvioutで確認する場合は以下を有効にする
%\usepackage[dviout]{graphicx,color}
% pdf化する場合は以下を有効にする
\usepackage[dvipdfmx]{graphicx,color}

%
% --------------------------------------------------------------------------
% 図表番号の後の:を削除
%
\makeatletter
\long\def\@makecaption#1#2{% #1=図表番号,#2=キャプション本文
\sbox\@tempboxa{#1 \hskip0.5zw #2}% 図表番号とキャプションの間のスペース 0.5zw
\ifdim \wd\@tempboxa >\hsize
#1 #2\par 
\else
\hb@xt@\hsize{\hfil\box\@tempboxa\hfil}
\fi}
\makeatother
% --------------------------------------------------------------------------

%
% --------------------------------------------------------------------------
% 下の該当する部分を書き換える
%
\氏名{本間 三暉}			%% 自分の氏名
\出席番号{35}					%% 出席番号
\研究室名{視覚情報処理研究室}			%% 研究室名
\指導教官{高橋 章}			%% 指導教員名
%
% --------------------------------------------------------------------------
% 研究題目等
%
% 研究題目が2行になるときは \\ で行送りできる
%
\発表番号{B\;--\;}
\研究題目{モーションキャプチャデバイスと画像処理を利用したバランス運動の解析}

% --------------------------------------------------------------------------
% アブストラクト
%
\アブストラクト{
モーションキャプチャデバイスmocopiを,運動解析に活用することを検討する.
そのため,竹馬や一輪車などの精密な重心移動や姿勢制御が必要な運動に関して測定を行う.
しかし,モーションキャプチャデバイスだけでは肘や膝などの関節の屈曲を正確に計測することができないため,
画像処理により骨格推定を組み合わせ,バランス運動時の重心や姿勢を解析する.
}
%
% --------------------------------------------------------------------------
% 本文開始
%
\begin{document}
\maketitle

%
% --------------------------------------------------------------------------
% 第1節
%
\section{研究背景・目的}
%
情報通信技術の急速な進歩により人工現実感,拡張現実感,複合現実感などの応用が広がっている.
感染症対策を契機にオンラインコミュニケーションも増加し,インターネット上の仮想共有空間であるメタバースが注目されている.
三次元の仮想空間で自分の分身となるアバターを自由に操作するには,
体の動きを計測する必要があり,画像処理による方法や専用デバイスを装着する方法などが試みられている.

本研究では市販のモーションキャプチャデバイスmocopiを,運動解析に活用することを検討する.
このデバイスは両手,両足,頭,腰の計6か所に小型センサを装着してリアルタイムに三次元計測を行うことができるが,肘や膝などの関節の屈曲を正確に計測することができない.
そこで画像処理による骨格推定を組み合わせ,一輪車や竹馬のような器具を使うバランス運動の動作解析を実現する.
これにより,身体が身体自信や使用器具などに隠れてしまう場合の骨格推定の精度低下の解消,三次元計測の精度の向上,計測速度のさらなる改善などの問題を解決することが出来る.
そして,スポーツや映像作品などの様々な分野で,使い方が限定されていたモーションキャプチャの応用範囲が広がることが期待できる.
%
% --------------------------------------------------------------------------
% 第2節
%
\section{研究内容・目的}
%
モーションキャプチャデバイスmocopiと画像処理による骨格推定を組み合わせ,
それぞれの測定方法の欠点となりうる部分を補いつつ,バランス運動を解析する.
%
% --------------------------------------------------------------------------
% 第2節 第1小節
%
\subsection{開発環境}
%
本研究では,開発環境としてmocopiとOpenPoseを使用する.

mocopi\cite{mocopi}とは,市販のモーションキャプチャデバイスで両手,両足,頭,腰の計6か所に小型センサを装着してリアルタイムに三次元計測を行うことができる.
mocopiのセンサはそれぞれ3つの自由度を持つ加速度センサと角度センサ

・mokopiに付いて記述


OpenPose\cite{openpose}とは,カーネギーメロン大学のCaoらによって発表された,18個のキーポイント(関節)とその関節をつなぐボーン(骨)を検出することができるオープンソースである.
OpenPoseは,正面からの画像だけでなく横からでも姿勢推定を行うことができる.
また,信頼度は低下するが遮蔽物により,見えない部位の推定も行うことができる.
本研究では,%画像左上を原点,画像上端を x 軸,画像左端を y 軸とし,図 1 に示す各 18 点の関節の x, y 座標と座標推定の信頼度を 1 フレームごとに取得し,時系列データとしてまとめる.

%
% --------------------------------------------------------------------------
% 第2節 第2小節
%
\subsection{3D座標の推定}
%

%
% --------------------------------------------------------------------------
% 第2節 第3小節
%
\subsection{(mocopi側の処理について)}
%
・思いついてない
%
% --------------------------------------------------------------------------
% 第2節 第4小節
%
\subsection{リアルタイム処理}
%
・一旦データを集めて集めたデータをあとで処理する形でやる.処理できそうなら,リアルタイム処理ができるか考える.
%
% --------------------------------------------------------------------------
% 第2節 第5小節
%
%\subsection{}
%

%
%
%
% --------------------------------------------------------------------------
% 第2節 第6小節
%
%\subsection{}
%

%
%
%
% --------------------------------------------------------------------------
% 第3節
%
\section{研究計画と進捗状況}
%

%
% --------------------------------------------------------------------------
% 第3節 第1小節
%
\subsection{研究の進め方}
%
mocopiとOpenPoseを用いて一輪車や竹馬などのものを用いるバランス運動の際の骨格を測定し,できる人とできない人の差を主に重心移動や姿勢の細かな違いについて解析する.
%
%
% --------------------------------------------------------------------------
% 第3節 第2小節
%
\subsection{研究方法や装置の概略}
%
本研究では,開発環境としてmocopiとOpenPoseを使用する.

mocopi\cite{mocopi}とは,市販のモーションキャプチャデバイスで両手,両足,頭,腰の計6か所に小型センサを装着してリアルタイムに三次元計測を行うことができる.
mocopiのセンサはそれぞれ3つの自由度を持つ加速度センサと角度センサ

・mokopiに付いて記述


OpenPose\cite{openpose}とは,カーネギーメロン大学のCaoらによって発表された,18個のキーポイント(関節)とその関節をつなぐボーン(骨)を検出することができるオープンソースである.
OpenPoseは,正面からの画像だけでなく横からでも姿勢推定を行うことができる.
また,信頼度は低下するが遮蔽物により,見えない部位の推定も行うことができる.
%

%
%
% --------------------------------------------------------------------------
% 第3節 第3小節
%
\subsection{進捗状況}
%
現在は,OpenPose による姿勢推定を進めている.水曜日までにもう少しできることを増やしておきたい.
%
% --------------------------------------------------------------------------
% 第3節 第4小節
%
%\subsection{}
%

%

%
%
% --------------------------------------------------------------------------
% 第3節
%
\section{まとめと今後の予定}
今は一旦竹馬だけだが,今後は一輪車やその他のものを用いるバランス運動についても行っていきたい.
%
% --------------------------------------------------------------------------
% 参考文献
%

\begin{thebibliography}{99}
  \small{
    \bibitem{mocopi}{
      モバイルモーションキャプチャー mocopi,https://www.sony.jp/mocopi/
    }
    \bibitem{openpose}{
      CAO,Zhe,et al.OpenPose: Realtime Multi-Person 2D Pose Estimation Using Part Affinity Fields. arXiv preprint arXiv:1812.08008. 2018.
    }
    \bibitem{takahashi}{
      高橋章, ``\TeX によるレポート作成'',
      電子制御工学科 第 3 学年前期学生実験テキスト, 2003
    }
  }
\end{thebibliography}


\end{document}
