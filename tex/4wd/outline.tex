% \documentclass[twoside,twocolumn]{ujarticle}
\documentclass[titlepage]{jarticle}
%\usepackage{type1cm}
\usepackage{outline-ec}
\usepackage{amsmath,amssymb,verbatim,ascmac,multicol}
\usepackage{tabularx}
% dvioutで確認する場合は以下を有効にする
%\usepackage[dviout]{graphicx,color}
% pdf化する場合は以下を有効にする
\usepackage[dvipdfmx]{graphicx,color}

%
% --------------------------------------------------------------------------
% 図表番号の後の:を削除
%
\makeatletter
\long\def\@makecaption#1#2{% #1=図表番号,#2=キャプション本文
\sbox\@tempboxa{#1 \hskip0.5zw #2}% 図表番号とキャプションの間のスペース 0.5zw
\ifdim \wd\@tempboxa >\hsize
#1 #2\par 
\else
\hb@xt@\hsize{\hfil\box\@tempboxa\hfil}
\fi}
\makeatother
% --------------------------------------------------------------------------

%
% --------------------------------------------------------------------------
% 下の該当する部分を書き換える
%
\氏名{本間 三暉}			%% 自分の氏名
\出席番号{35}					%% 出席番号
\研究室名{視覚情報処理研究室}			%% 研究室名
\指導教官{高橋 章}			%% 指導教員名
%
% --------------------------------------------------------------------------
% 研究題目等
%
% 研究題目が2行になるときは \\ で行送りできる
%
\発表番号{B\;--\;}
\研究題目{モーションキャプチャデバイスと画像処理を利用したバランス運動の解析}

% --------------------------------------------------------------------------
% アブストラクト
%
\アブストラクト{
<<<<<<< HEAD
a
=======
モーションキャプチャデバイスmocopiでは表現しきれない部分を画像処理を用いて補助することを検討する.
mocopiでは関節部などが正確に計測できないことがあるため,竹馬や一輪車などの精密な重心移動や姿勢制御が必要な運動に関して測定を行い,
画像処理による骨格推定を組み合わせ,バランス運動時の重心や姿勢を解析することによって,mocopiの測定精度を上げることを目指す.
>>>>>>> 2db553debe738d408b1081a6528e1cd018aa4c13
}
%
% --------------------------------------------------------------------------
% 本文開始
%
\begin{document}
\maketitle

%
% --------------------------------------------------------------------------
% 第1節
%
\section{研究背景・目的}
%
情報通信技術の急速な進歩により人工現実感,拡張現実感,複合現実感などの応用が広がっている.
感染症対策を契機にオンラインコミュニケーションも増加し,インターネット上の仮想共有空間であるメタバースが注目されている.
三次元の仮想空間で自分の分身となるアバターを自由に操作するには,
体の動きを計測する必要があり,画像処理による方法や専用デバイスを装着する方法などが試みられている.
<<<<<<< HEAD
特に画像処理による方法で三次元の情報を取得するためには複数台のカメラを用いる方法があるが,狭い室内であるなどの場所の制約や,
限られた予算の中で実装したいという予算の制約などによってこの方法を取るのが難しい場合がある.

本研究ではカメラ1台で三次元骨格推定ができる現行の方法について比較し,
組み込みPCでの実装やリアルタイム処理などの高速化,
オクルージョンへの対応などの高精度化を目指す.
% 本研究では市販のモーションキャプチャデバイスmocopiを,運動解析に活用することを検討する.
% このデバイスは両手,両足,頭,腰の計6か所に小型センサを装着してリアルタイムに三次元計測を行うことができるが,肘や膝などの関節の屈曲を正確に計測することができない.
% そこで画像処理による骨格推定を組み合わせ,一輪車や竹馬のような器具を使うバランス運動の動作解析を実現する.
% これにより,身体が身体自信や使用器具などに隠れてしまう場合の骨格推定の精度低下の解消,三次元計測の精度の向上,計測速度のさらなる改善などの問題を解決することが出来る.
% そして、スポーツや映像作品などの様々な分野で,使い方が限定されていたモーションキャプチャの応用範囲が広がることが期待できる.
=======

本研究では市販のモーションキャプチャデバイスmocopiを運動解析に活用し,測定精度を上げることを目指す.
このデバイスは両手,両足,頭,腰の計6か所に小型センサを装着してリアルタイムに三次元計測を行うものだが,肘や膝などの関節の屈曲を正確に計測することができない.
そこで画像処理による骨格推定を組み合わせ,体を大きく動かす運動を測定しmocopiの精度向上を目指す.

これにより,mocopiの問題点である肘や膝などの関節の屈曲を正確に計測することができないことだけでなく,画像処理の身体が身体自信や使用器具などに隠れてしまう場合の骨格推定の精度低下の解消,三次元計測の精度の向上,計測速度のさらなる改善などの問題を解決することが出来る.
そして,スポーツや映像作品などの様々な分野で使い方が限定されていた,モーションキャプチャの応用範囲が広がることが期待できる.
>>>>>>> 2db553debe738d408b1081a6528e1cd018aa4c13
%
% --------------------------------------------------------------------------
% 第2節
%
\section{研究内容}
%
<<<<<<< HEAD

=======
モーションキャプチャデバイスmocopiと画像処理による骨格推定を組み合わせ,
それぞれの測定方法の欠点となりうる部分を補いつつ,体を大きく動かす運動を解析する.
>>>>>>> 2db553debe738d408b1081a6528e1cd018aa4c13
%
% --------------------------------------------------------------------------
% 第2節 第1小節
%
\subsection{開発環境}
%
<<<<<<< HEAD
本研究では,開発環境としてOpenPoseを使用する.
=======
本研究では,開発環境としてmocopiとOpenPoseを使用する.

mocopi\cite{mocopi}とは,市販のモーションキャプチャデバイスで両手,両足,頭,腰の計6か所に小型センサを装着してリアルタイムに三次元計測を行うことができる.
mocopiのセンサはそれぞれ3つの自由度を持つ加速度センサと角度センサ

・mokopiに付いて記述


OpenPose\cite{openpose}とは,カーネギーメロン大学のCaoらによって発表された,18個のキーポイント(関節)とその関節をつなぐボーン(骨)を検出することができるオープンソースである.
OpenPoseは,正面からの画像だけでなく横からでも姿勢推定を行うことができる.
また,信頼度は低下するが遮蔽物により,見えない部位の推定も行うことができる.
本研究では,%画像左上を原点,画像上端を x 軸,画像左端を y 軸とし,図 1 に示す各 18 点の関節の x, y 座標と座標推定の信頼度を 1 フレームごとに取得し,時系列データとしてまとめる.

>>>>>>> 2db553debe738d408b1081a6528e1cd018aa4c13
%
% --------------------------------------------------------------------------
% 第2節 第2小節
%
\subsection{3D座標の推定}
%

%
% --------------------------------------------------------------------------
% 第2節 第3小節
%
\subsection{(mocopi側の処理について)}
%
・思いついてない
%
% --------------------------------------------------------------------------
% 第2節 第4小節
%
\subsection{リアルタイム処理}
%
・一旦データを集めて集めたデータをあとで処理する形でやる.処理できそうなら,リアルタイム処理ができるか考える.
%
% --------------------------------------------------------------------------
% 第2節 第5小節
%
%\subsection{}
%

%
%
%
% --------------------------------------------------------------------------
% 第2節 第6小節
%
%\subsection{}
%

%
%
%
% --------------------------------------------------------------------------
% 第3節
%
\section{研究計画と進捗状況}
%

%
% --------------------------------------------------------------------------
% 第3節 第1小節
%
\subsection{研究の進め方}
%
mocopiとOpenPoseを用いて体を大きく動かす学校体操,空手の演武,ヨガなどの運動を行った際の骨格を測定し,関節部分が正しく測定できるようにする.
%
%
% --------------------------------------------------------------------------
% 第3節 第2小節
%
\subsection{研究方法や装置の概略}
%
本研究では,開発環境としてmocopiとOpenPoseを使用する.

mocopi\cite{mocopi}とは,市販のモーションキャプチャデバイスで両手,両足,頭,腰の計6か所に小型センサを装着してリアルタイムに三次元計測を行うことができる.
mocopiのセンサはそれぞれ3つの自由度を持つ加速度センサと角度センサ

・mokopiに付いて記述


OpenPose\cite{openpose}とは,カーネギーメロン大学のCaoらによって発表された,18個のキーポイント(関節)とその関節をつなぐボーン(骨)を検出することができるオープンソースである.
OpenPoseは,正面からの画像だけでなく横からでも姿勢推定を行うことができる.
また,信頼度は低下するが遮蔽物により,見えない部位の推定も行うことができる.
%

%
%
% --------------------------------------------------------------------------
% 第3節 第3小節
%
\subsection{進捗状況}
%
現在は,OpenPoseによる姿勢推定を進めている.
%
% --------------------------------------------------------------------------
% 第3節 第4小節
%
%\subsection{}
%

%

%
%
% --------------------------------------------------------------------------
% 第3節
%
\section{まとめと今後の予定}
学校体操や空手の型などの,体を大きく動かし,関節の位置や体の相対関係が正しく測定できるようにする.
%
% --------------------------------------------------------------------------
% 参考文献
%

\begin{thebibliography}{99}
  \small{
    \bibitem{mocopi}{
      モバイルモーションキャプチャー mocopi,https://www.sony.jp/mocopi/
    }
    \bibitem{openpose}{
      CAO,Zhe,et al.OpenPose: Realtime Multi-Person 2D Pose Estimation Using Part Affinity Fields. arXiv preprint arXiv:1812.08008. 2018.
    }
  }
\end{thebibliography}


\end{document}
