\documentclass[dvipdfmx]{jsarticle}
\usepackage{okumacro}

\begin{document}

\title{受験報告書}
\author{} % 書かなくて良い.名前欄
\date{}   % 書かなくて良い.作成日時
\maketitle
% 例としてR4実施の専攻科の学力試験を受けた想定でそれっぽく記述しておく.参考にすると良い.
% テンプレートなので最低限しか記述しないため,後輩のことを思って各自付け足すと良い.
\section{受験先}
\begin{description}
  \item[受験学科:] 長岡技大・情報経営システム分野
    % 基本的には「大学名・学科名」のような形式が好ましいが,それっぽくなるように各自で工夫してほしい.
  \item[試験区分:] 学力
    % 学力 or 推薦 他にも何かあったらそれを書くと良い.
  \item[受験結果:] 合格
    % 合格 or 不合格 ないとは思うが他にもなにかあったら書くと良い.
  \item[願書受付:] 5/8(月) 17:00まで
    % これは書かなくてもいいかもしれない.親切な人は書くと良い.
  \item[試験日程:] 6/11(土)
    % 2日間ある場合は両日書くべし.
    % 来年以降試験日程の組み立てなどに使われることを意識して書くべし.
  \item[合格発表:] 7/13(木)
    % これは書かなくてもいいかもしれない.親切な人は書くと良い.
\end{description}

\section{交通手段・宿泊先}
% 専攻科は書かなくても良さそうな雰囲気がある.

\subsection{交通手段}
% 試験前日から泊まり,宿泊先から移動する場合や試験が二日間ある場合は
% \subsubsectionなどを用いて分けて書くと良い.
% 公共交通機関を使った場合は,単に乗り物の名前ではなく具体的な本線などの名前を書くことが望ましい.
自宅 $\rightarrow$ (自転車) $\rightarrow$ 新潟駅 $\rightarrow$ (信越本線)
$\rightarrow$ 長岡駅 $\rightarrow$ (成願寺行き) $\rightarrow$ 長岡高専

寝坊するのが怖かったので前日に移動を済ませました.

\subsection{宿泊先}
\subsubsection*{}
% 宿泊先の立地などを中心にかけると良い.
% 失敗した点などがある場合はそれもかけると良い.


\section{受験者情報}
% もしかするとこれ書かなくていいかも.あともう少しマシなフォーマットありそう
\begin{description}
  \item[  席次:] 1年次40/40位,2年次30/41位,3年次24/39位,4年次9/40位
  \item[TOEIC:] 325点
  \item[ 併願校:] 長岡技大,電通大,農工大,東工大
\end{description}

もともと横国と筑波も受けるつもりでしたが,英語の点数換算に使われるTOEICがお粗末過ぎたので断念しました.
TOEICが830あればだいたいどこでも満点換算されるはずです.

長岡技大,電通大,筑波,横国,農工大,東工大は個人的に最高の受験計画だと思うので東工大が第一志望の人は狙ってみるのもいいかもしれません.

\section{願書提出}
第0回戦とも言える願書提出です.いくつか注意点があるので作りました.

\subsection{インターネット出願}
周りの話を聞いた感じ専攻科でもやってるみたいですが,専攻科とは違い証明写真をアップロード
しないで中断することができません.
これをしないことには何も始まらないので早めにやるようにしてください.

証明写真は基本面接をするときの服装であるスーツだと思います.なので少なくとも4年から5年にかけての
春休み中にはスーツが手元にある状態にしておきましょう.私を含め,スーツを後回しにした結果痛い目を見ている人が何人かいました.

\subsection{成績証明書}
春休み明けすぐくらいに学生課で印刷できるようになります.早めに行きましょう.
願書提出は速達だからGWでいいかと後回しにした挙げ句,
GW中に「成績証明書ってどこから入手?」と聞いてきた人がいました.
もちろんGW中は学生課が開いてないので印刷できず,提出期限日に直接技大まで提出しに行ってました.

印刷可能になったらとりあえず受ける大学分取りに行きましょう.

\subsection{死亡嘲笑}
志望調書です.
研究内容が絡む内容もあるので,早めに指導教員と相談して春休み中には``\textgt{研究テーマの確定}''と``\textgt{志望調書の第一稿完成}''までは終わらせておくことをとても強くおすすめします.
少なくとも私は病みました.

そうでなくとも,休み明けから動き出すと所属研究室にもよるとは思いますが,4月中はまともに勉強できないと思ってください.

\subsection{検定料}
どの大学も基本3万円です.
1校受けるだけで結構な額になるので親か自分のお財布とよくよく相談してください.


\section{試験内容}
% 試験日程を書くのも良いかもしれない.
% 試験の対策として何を行ったか,何をするのがいいか,当日の試験の範囲や手応えなどを書けると良い
% 面接試験があった人は何を聞かれたかも書くと良い
制御から行くとすれば基本的に``\textgt{電気電子情報分野}''か``\textgt{情報経営システム分野}''のどっちかだと思います.
``\textgt{電気電子情報分野}''の話は満足にできないと思うのでこの節は飛ばすことを推奨します.
\subsection{前提情報}
``\textgt{情報経営システム分野}''は``\textgt{電気電子情報分野}''の1/2ほどの募集人数であり,専門科目ですべての学科を選ぶことができます.

``\textgt{電気電子情報分野}''を選ぶ場合制御は推薦の基準こそ満たせるものの,
\ruby{他の学科}{電気電子システム}と比べ\ruby{高得点}{苦労せず満点}を取れる科目が少ないので``\textgt{電気電子情報分野}''の枠を\ruby{他の学科}{電気電子システム}やその他多くの高専の人達に多く取られることになり,
厳しい戦いを強いられることになります.

しかし``\textgt{情報経営システム分野}''は競争相手が\ruby{群馬}{現帝国}高専の人達と制御だけなので
比較的入りやすいです.人数差という話では,``\textgt{電気電子情報分野}''の受験者で441くらい大きな講義室が埋まったのに対し,``\textgt{情報経営システム分野}'',物質,環境が一つの講義室でした.(という話を先輩から聞きました.)

\subsection{国語}
配点が100点なので基本的に誰も対策してません.
多分まじで日本語が不自由な人を落とすための試験です.
国語の対策をするくらいなら専門科目で満点を取れるようにしましょう.

\subsection{英語}
英語VDを真面目に受けるだけで大丈夫です.ちゃんと先生の話を聞きましょう.
TOEIC470取ってる人も免除申請だけ出して受けに行きましょう.
ただ枠を圧迫するので技大を受けない人には受講しないでほしいといったほうがいいと思います.

\subsection{数学}
これはどの大学にも言えることですが,まずは過去問や体験記などを見て出る範囲の``\textgt{研究}''をしましょう.
大学によって微分方程式や確率の範囲が出ないことがあります.
出ない範囲の勉強をしても無駄なのでこれが一番大事です.
また,出題傾向が大きく変わっている場合があるので,コロナ前の過去問を信用しすぎるのもやめましょう.
少しでも新しい情報のほうが信憑性が高いです.

基本的には以下の参考書を用いて勉強しました.
\begin{itemize}
  \item 大学編入のための数学問題集
  \item 編入数学徹底研究
  \item 編入数学過去問特訓
  \item 数学/徹底演習
\end{itemize}
周りは編入数学徹底研究を何周もしてました.
正直編入数学徹底研究か大学編入のための数学問題集のどっちかがあれば技大,専攻科は余裕だと思います.

受験してそうな先輩や早めに勉強を初めてそうな同級生に中身を見せてもらったりして自分に合ってそうな方を選ぶといいと思います.

\subsection{専門科目}



\section{その他}
% 感想や後輩に対する遺言,アドバイスなどを書くと良い.
% 特に東大,京大,東工大などほとんどの人が受けない学校の場合は数少ない情報源となるのでできるだけ詳しく書くこと
\begin{itemize}
  \item 去年の先輩は数学で満点を取っていても落ちていたので不安でしたが受かって安心しました.
  \item 4月はとても忙しく,まともに勉強ができないスケジュールを課せられる場合もあるので
        春休みまでに一通りの勉強を終わらせる見通しを立てましょう.
  \item 募集要項が出る前でも昨年の募集要項等を見て準備できるものは準備しましょう.休み明け多少楽ができます.
\end{itemize}
\end{document}
