\documentclass[dvipdfmx]{jsarticle}
\begin{document}

\title{受験報告書}
\author{} % 書かなくて良い.名前欄
\date{}   % 書かなくて良い.作成日時
\maketitle
% 例としてR4実施の専攻科の学力試験を受けた想定でそれっぽく記述しておく.参考にすると良い.
% テンプレートなので最低限しか記述しないため,後輩のことを思って各自付け足すと良い.
\section{受験先}
\begin{description}
  \item[受験学科:] 長岡高専専攻科・電子機械システム工学専攻
    % 基本的には「大学名・学科名」のような形式が好ましいが,それっぽくなるように各自で工夫してほしい.
  \item[試験区分:] 学力
    % 学力 or 推薦 他にも何かあったらそれを書くと良い.
  \item[受験結果:] 合格
    % 合格 or 不合格 ないとは思うが他にもなにかあったら書くと良い.
    % \item[願書受付:]
    % これは書かなくてもいいかもしれない.親切な人は書くと良い.
  \item[試験日程:] 6/11(土)
    % 2日間ある場合は両日書くべし.
    % 来年以降試験日程の組み立てなどに使われることを意識して書くべし.
    % \item[合格発表:] 6/22(水)
    % これは書かなくてもいいかもしれない.親切な人は書くと良い.
\end{description}

\section{交通手段・宿泊先}
% 専攻科は書かなくても良さそうな雰囲気がある.

\subsection{交通手段}
% 試験前日から泊まり,宿泊先から移動する場合や試験が二日間ある場合は
% \subsubsectionなどを用いて分けて書くと良い.
% 公共交通機関を使った場合は,単に乗り物の名前ではなく具体的な本線などの名前を書くことが望ましい.
自宅 $\rightarrow$ (自転車) $\rightarrow$ 新潟駅 $\rightarrow$ (信越本線)
$\rightarrow$ 長岡駅 $\rightarrow$ (成願寺行き) $\rightarrow$ 長岡高専

寝坊するのが怖かったので前日に移動を済ませました.

\subsection{宿泊先}
\subsubsection*{視覚情報処理研究室}
% 宿泊先の立地などを中心にかけると良い.
% 失敗した点などがある場合はそれもかけると良い.
試験会場まで歩いて5分未満の研究室です.
水道,冷蔵庫,電子レンジ,コーヒーメーカーなどが揃っているのでとても快適でした.
周辺には売店もあるので食事の面では不自由しないと思います.

\section{受験者情報}
% もしかするとこれ書かなくていいかも.あともう少しマシなフォーマットありそう
\begin{description}
  \item[  席次:] 1年次40位,2年次40位,3年次40位,4年次40位
  \item[TOEIC:] 1000点
  \item[ 併願校:] 専攻科,長岡技大,横国
\end{description}

\section{試験内容}
% 試験日程を書くのも良いかもしれない.
% 試験の対策として何を行ったか,何をするのがいいか,当日の試験の範囲や手応えなどを書けると良い
% 面接試験があった人は何を聞かれたかも書くと良い
\subsection{調査書}
募集要項を見ると調査書の内容が100点分の割合を占めています.
おそらく4年次の席次が関係していると考えられるので日々の定期テストにしっかり取り組みましょう.


\subsection{数学}
専攻科の数学はそこまで難易度は高くないと思います.
逆に言うとここではほとんどの人が10割取ると思うのでケアレスミスには気をつけたいところです.

\subsubsection*{基礎数学の問題}
底の変換,数列,微積,逆行列を求める問題などが出ます.
取れなきゃ恥ずかしいのでここは死んでも取るようにしましょう.

\subsubsection*{二階非斉次線形微分方程式}
4年の応用数学の内容です.
授業をしっかり聞いて解けるようにしておくといいでしょう.
私はここで計算ミスをしてしまい出てきた解を代入しても与式にならなくて時間を取られました.

\subsubsection*{空間ベクトル}
長岡の高専生がおそらく一番苦手とする分野です.2,3年の復習をしっかりとしましょう.

\subsubsection*{二変数関数の極値問題,重積分}
二変数関数の極値問題ですが,計算ミスでヘッシアンが0になってしまい面倒なことになりました.
そのせいで時間が足りなくて重積分は解けてません.


\subsection{専門科目}
昨年から試験範囲が変わり電磁気が出なくなったのを許してません.


\section{その他}
% 感想や後輩に対する遺言,アドバイスなどを書くと良い.
% 特に東大,京大,東工大などほとんどの人が受けない学校の場合は数少ない情報源となるのでできるだけ詳しく書くこと
\begin{itemize}
  \item 定期試験と違って前日に過去問を見るだけでは不十分なので気をつけましょう.
  \item 去年の先輩は数学で満点を取っていても落ちていたので不安でしたが受かって安心しました.
  \item 4月はとても忙しく,まともに勉強ができないスケジュールを課せられる場合もあるので
        春休みまでに一通りの勉強を終わらせる見通しを立てましょう.
  \item 募集要項が出る前でも昨年の募集要項等を見て準備できるものは準備しましょう.休み明け多少楽ができます.
\end{itemize}
\end{document}
